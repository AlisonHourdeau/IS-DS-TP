\documentclass[]{article}
\usepackage{lmodern}
\usepackage{amssymb,amsmath}
\usepackage{ifxetex,ifluatex}
\usepackage{fixltx2e} % provides \textsubscript
\ifnum 0\ifxetex 1\fi\ifluatex 1\fi=0 % if pdftex
  \usepackage[T1]{fontenc}
  \usepackage[utf8]{inputenc}
\else % if luatex or xelatex
  \ifxetex
    \usepackage{mathspec}
  \else
    \usepackage{fontspec}
  \fi
  \defaultfontfeatures{Ligatures=TeX,Scale=MatchLowercase}
\fi
% use upquote if available, for straight quotes in verbatim environments
\IfFileExists{upquote.sty}{\usepackage{upquote}}{}
% use microtype if available
\IfFileExists{microtype.sty}{%
\usepackage{microtype}
\UseMicrotypeSet[protrusion]{basicmath} % disable protrusion for tt fonts
}{}
\usepackage[margin=1in]{geometry}
\usepackage{hyperref}
\hypersetup{unicode=true,
            pdftitle={TP1 - CA},
            pdfborder={0 0 0},
            breaklinks=true}
\urlstyle{same}  % don't use monospace font for urls
\usepackage{color}
\usepackage{fancyvrb}
\newcommand{\VerbBar}{|}
\newcommand{\VERB}{\Verb[commandchars=\\\{\}]}
\DefineVerbatimEnvironment{Highlighting}{Verbatim}{commandchars=\\\{\}}
% Add ',fontsize=\small' for more characters per line
\usepackage{framed}
\definecolor{shadecolor}{RGB}{248,248,248}
\newenvironment{Shaded}{\begin{snugshade}}{\end{snugshade}}
\newcommand{\AlertTok}[1]{\textcolor[rgb]{0.94,0.16,0.16}{#1}}
\newcommand{\AnnotationTok}[1]{\textcolor[rgb]{0.56,0.35,0.01}{\textbf{\textit{#1}}}}
\newcommand{\AttributeTok}[1]{\textcolor[rgb]{0.77,0.63,0.00}{#1}}
\newcommand{\BaseNTok}[1]{\textcolor[rgb]{0.00,0.00,0.81}{#1}}
\newcommand{\BuiltInTok}[1]{#1}
\newcommand{\CharTok}[1]{\textcolor[rgb]{0.31,0.60,0.02}{#1}}
\newcommand{\CommentTok}[1]{\textcolor[rgb]{0.56,0.35,0.01}{\textit{#1}}}
\newcommand{\CommentVarTok}[1]{\textcolor[rgb]{0.56,0.35,0.01}{\textbf{\textit{#1}}}}
\newcommand{\ConstantTok}[1]{\textcolor[rgb]{0.00,0.00,0.00}{#1}}
\newcommand{\ControlFlowTok}[1]{\textcolor[rgb]{0.13,0.29,0.53}{\textbf{#1}}}
\newcommand{\DataTypeTok}[1]{\textcolor[rgb]{0.13,0.29,0.53}{#1}}
\newcommand{\DecValTok}[1]{\textcolor[rgb]{0.00,0.00,0.81}{#1}}
\newcommand{\DocumentationTok}[1]{\textcolor[rgb]{0.56,0.35,0.01}{\textbf{\textit{#1}}}}
\newcommand{\ErrorTok}[1]{\textcolor[rgb]{0.64,0.00,0.00}{\textbf{#1}}}
\newcommand{\ExtensionTok}[1]{#1}
\newcommand{\FloatTok}[1]{\textcolor[rgb]{0.00,0.00,0.81}{#1}}
\newcommand{\FunctionTok}[1]{\textcolor[rgb]{0.00,0.00,0.00}{#1}}
\newcommand{\ImportTok}[1]{#1}
\newcommand{\InformationTok}[1]{\textcolor[rgb]{0.56,0.35,0.01}{\textbf{\textit{#1}}}}
\newcommand{\KeywordTok}[1]{\textcolor[rgb]{0.13,0.29,0.53}{\textbf{#1}}}
\newcommand{\NormalTok}[1]{#1}
\newcommand{\OperatorTok}[1]{\textcolor[rgb]{0.81,0.36,0.00}{\textbf{#1}}}
\newcommand{\OtherTok}[1]{\textcolor[rgb]{0.56,0.35,0.01}{#1}}
\newcommand{\PreprocessorTok}[1]{\textcolor[rgb]{0.56,0.35,0.01}{\textit{#1}}}
\newcommand{\RegionMarkerTok}[1]{#1}
\newcommand{\SpecialCharTok}[1]{\textcolor[rgb]{0.00,0.00,0.00}{#1}}
\newcommand{\SpecialStringTok}[1]{\textcolor[rgb]{0.31,0.60,0.02}{#1}}
\newcommand{\StringTok}[1]{\textcolor[rgb]{0.31,0.60,0.02}{#1}}
\newcommand{\VariableTok}[1]{\textcolor[rgb]{0.00,0.00,0.00}{#1}}
\newcommand{\VerbatimStringTok}[1]{\textcolor[rgb]{0.31,0.60,0.02}{#1}}
\newcommand{\WarningTok}[1]{\textcolor[rgb]{0.56,0.35,0.01}{\textbf{\textit{#1}}}}
\usepackage{graphicx,grffile}
\makeatletter
\def\maxwidth{\ifdim\Gin@nat@width>\linewidth\linewidth\else\Gin@nat@width\fi}
\def\maxheight{\ifdim\Gin@nat@height>\textheight\textheight\else\Gin@nat@height\fi}
\makeatother
% Scale images if necessary, so that they will not overflow the page
% margins by default, and it is still possible to overwrite the defaults
% using explicit options in \includegraphics[width, height, ...]{}
\setkeys{Gin}{width=\maxwidth,height=\maxheight,keepaspectratio}
\IfFileExists{parskip.sty}{%
\usepackage{parskip}
}{% else
\setlength{\parindent}{0pt}
\setlength{\parskip}{6pt plus 2pt minus 1pt}
}
\setlength{\emergencystretch}{3em}  % prevent overfull lines
\providecommand{\tightlist}{%
  \setlength{\itemsep}{0pt}\setlength{\parskip}{0pt}}
\setcounter{secnumdepth}{0}
% Redefines (sub)paragraphs to behave more like sections
\ifx\paragraph\undefined\else
\let\oldparagraph\paragraph
\renewcommand{\paragraph}[1]{\oldparagraph{#1}\mbox{}}
\fi
\ifx\subparagraph\undefined\else
\let\oldsubparagraph\subparagraph
\renewcommand{\subparagraph}[1]{\oldsubparagraph{#1}\mbox{}}
\fi

%%% Use protect on footnotes to avoid problems with footnotes in titles
\let\rmarkdownfootnote\footnote%
\def\footnote{\protect\rmarkdownfootnote}

%%% Change title format to be more compact
\usepackage{titling}

% Create subtitle command for use in maketitle
\providecommand{\subtitle}[1]{
  \posttitle{
    \begin{center}\large#1\end{center}
    }
}

\setlength{\droptitle}{-2em}

  \title{TP1 - CA}
    \pretitle{\vspace{\droptitle}\centering\huge}
  \posttitle{\par}
    \author{}
    \preauthor{}\postauthor{}
    \date{}
    \predate{}\postdate{}
  

\begin{document}
\maketitle

\hypertarget{tp1-distance-similarituxe9-et-inerties}{%
\section{TP1 : Distance, similarité et
inerties}\label{tp1-distance-similarituxe9-et-inerties}}

\begin{Shaded}
\begin{Highlighting}[]
\NormalTok{data <-}\StringTok{ }\KeywordTok{matrix}\NormalTok{(}\KeywordTok{c}\NormalTok{(}\DecValTok{2}\NormalTok{,}\DecValTok{2}\NormalTok{,}\FloatTok{7.5}\NormalTok{,}\DecValTok{4}\NormalTok{,}\DecValTok{3}\NormalTok{,}\DecValTok{3}\NormalTok{,}\FloatTok{0.5}\NormalTok{,}\DecValTok{5}\NormalTok{,}\DecValTok{6}\NormalTok{,}\DecValTok{4}\NormalTok{,}\FloatTok{1.5}\NormalTok{,}\DecValTok{7}\NormalTok{),}\DataTypeTok{byrow=}\OtherTok{TRUE}\NormalTok{,}\DataTypeTok{ncol=}\DecValTok{2}\NormalTok{)}
\KeywordTok{colnames}\NormalTok{(data) <-}\StringTok{ }\KeywordTok{c}\NormalTok{(}\StringTok{"X1"}\NormalTok{,}\StringTok{"X2"}\NormalTok{)}
\KeywordTok{rownames}\NormalTok{(data) <-}\StringTok{ }\KeywordTok{c}\NormalTok{(}\StringTok{"w1"}\NormalTok{,}\StringTok{"w2"}\NormalTok{,}\StringTok{"w3"}\NormalTok{,}\StringTok{"w4"}\NormalTok{,}\StringTok{"w5"}\NormalTok{,}\StringTok{"w6"}\NormalTok{)}
\NormalTok{data}
\end{Highlighting}
\end{Shaded}

\begin{verbatim}
##     X1 X2
## w1 2.0  2
## w2 7.5  4
## w3 3.0  3
## w4 0.5  5
## w5 6.0  4
## w6 1.5  7
\end{verbatim}

\hypertarget{question-1-contruire-les-matrices-des-distances-entre-les-individus}{%
\subsection{Question 1 : Contruire les matrices des distances entre les
individus}\label{question-1-contruire-les-matrices-des-distances-entre-les-individus}}

\hypertarget{en-utilisant-la-distance-de-manhattan}{%
\subsubsection{En utilisant la distance de
manhattan}\label{en-utilisant-la-distance-de-manhattan}}

\begin{Shaded}
\begin{Highlighting}[]
\NormalTok{dist_manhattan=}\KeywordTok{dist}\NormalTok{(data,}\DataTypeTok{method=}\StringTok{"manhattan"}\NormalTok{)}
\KeywordTok{as.matrix}\NormalTok{(dist_manhattan)}
\end{Highlighting}
\end{Shaded}

\begin{verbatim}
##     w1  w2  w3  w4  w5  w6
## w1 0.0 7.5 2.0 4.5 6.0 5.5
## w2 7.5 0.0 5.5 8.0 1.5 9.0
## w3 2.0 5.5 0.0 4.5 4.0 5.5
## w4 4.5 8.0 4.5 0.0 6.5 3.0
## w5 6.0 1.5 4.0 6.5 0.0 7.5
## w6 5.5 9.0 5.5 3.0 7.5 0.0
\end{verbatim}

\hypertarget{en-utilisant-la-distance-euclidienne}{%
\subsubsection{En utilisant la distance
euclidienne}\label{en-utilisant-la-distance-euclidienne}}

\begin{Shaded}
\begin{Highlighting}[]
\NormalTok{dist_euclidienne=}\KeywordTok{dist}\NormalTok{(data,}\DataTypeTok{method=}\StringTok{"euclidean"}\NormalTok{)}
\KeywordTok{as.matrix}\NormalTok{(dist_euclidienne)}
\end{Highlighting}
\end{Shaded}

\begin{verbatim}
##          w1       w2       w3       w4       w5       w6
## w1 0.000000 5.852350 1.414214 3.354102 4.472136 5.024938
## w2 5.852350 0.000000 4.609772 7.071068 1.500000 6.708204
## w3 1.414214 4.609772 0.000000 3.201562 3.162278 4.272002
## w4 3.354102 7.071068 3.201562 0.000000 5.590170 2.236068
## w5 4.472136 1.500000 3.162278 5.590170 0.000000 5.408327
## w6 5.024938 6.708204 4.272002 2.236068 5.408327 0.000000
\end{verbatim}

\hypertarget{comparer-les-ruxe9sultats}{%
\subsubsection{Comparer les résultats}\label{comparer-les-ruxe9sultats}}

Bien que les valeurs soient différentes, l'ordre de grandeur des
distances entre les individus ici reste inchangé quelque soit la
distance appliquée soit la distance appliquée à notre jeu de données.

\hypertarget{question-2-calcul-des-barycentres-de-la-partition-w1w3w4w2w5w6}{%
\subsection{Question 2 : Calcul des barycentres de la partition
\{\{w1,w3\};w4;\{w2,w5\};w6\}}\label{question-2-calcul-des-barycentres-de-la-partition-w1w3w4w2w5w6}}

\begin{Shaded}
\begin{Highlighting}[]
\NormalTok{barycentre<-}\ControlFlowTok{function}\NormalTok{(vect)\{}
  \KeywordTok{return}\NormalTok{(}\KeywordTok{apply}\NormalTok{(vect,}\DecValTok{2}\NormalTok{,mean))}
\NormalTok{\}}
\end{Highlighting}
\end{Shaded}

On applique cette fonction à la partition

\begin{Shaded}
\begin{Highlighting}[]
\NormalTok{G1_}\DecValTok{3}\NormalTok{<-}\KeywordTok{barycentre}\NormalTok{(data[}\KeywordTok{c}\NormalTok{(}\DecValTok{1}\NormalTok{,}\DecValTok{3}\NormalTok{),])}
\NormalTok{G4<-data[}\DecValTok{4}\NormalTok{,]}
\NormalTok{G2_}\DecValTok{5}\NormalTok{<-}\KeywordTok{barycentre}\NormalTok{(data[}\KeywordTok{c}\NormalTok{(}\DecValTok{2}\NormalTok{,}\DecValTok{5}\NormalTok{),])}
\NormalTok{G6<-data[}\DecValTok{6}\NormalTok{,]}
\NormalTok{barycentres <-}\StringTok{ }\KeywordTok{rbind}\NormalTok{(G1_}\DecValTok{3}\NormalTok{,G4,G2_}\DecValTok{5}\NormalTok{,G6)}
\KeywordTok{as.matrix}\NormalTok{(barycentres)}
\end{Highlighting}
\end{Shaded}

\begin{verbatim}
##        X1  X2
## G1_3 2.50 2.5
## G4   0.50 5.0
## G2_5 6.75 4.0
## G6   1.50 7.0
\end{verbatim}

\hypertarget{question-3-calcul-des-distances-des-points-avec-le-barycentre}{%
\subsection{Question 3 : Calcul des distances des points avec le
barycentre}\label{question-3-calcul-des-distances-des-points-avec-le-barycentre}}

\begin{Shaded}
\begin{Highlighting}[]
\NormalTok{distance<-}\ControlFlowTok{function}\NormalTok{(vect)\{}
\NormalTok{  bar<-}\KeywordTok{barycentre}\NormalTok{(vect)}
\NormalTok{  vect_}\DecValTok{2}\NormalTok{<-}\KeywordTok{rbind}\NormalTok{(vect,bar)}
\NormalTok{  z<-}\KeywordTok{dist}\NormalTok{(vect_}\DecValTok{2}\NormalTok{,}\DataTypeTok{method =} \StringTok{"euclidean"}\NormalTok{)}
  \KeywordTok{return}\NormalTok{(}\KeywordTok{as.matrix}\NormalTok{(z))}
\NormalTok{\}}
\end{Highlighting}
\end{Shaded}

\begin{Shaded}
\begin{Highlighting}[]
\NormalTok{d1<-}\KeywordTok{distance}\NormalTok{(data[}\KeywordTok{c}\NormalTok{(}\DecValTok{1}\NormalTok{,}\DecValTok{3}\NormalTok{),])}
\NormalTok{d1}
\end{Highlighting}
\end{Shaded}

\begin{verbatim}
##            w1        w3       bar
## w1  0.0000000 1.4142136 0.7071068
## w3  1.4142136 0.0000000 0.7071068
## bar 0.7071068 0.7071068 0.0000000
\end{verbatim}

Interprétation : 1,41 : distance entre w1 et w3 et la distance entre w3
et w1. Car c'est symétrique 0,71 : distance entre le barycentre et w1.

\begin{Shaded}
\begin{Highlighting}[]
\NormalTok{d2<-}\KeywordTok{distance}\NormalTok{(data[}\KeywordTok{c}\NormalTok{(}\DecValTok{2}\NormalTok{,}\DecValTok{5}\NormalTok{),])}
\NormalTok{d2}
\end{Highlighting}
\end{Shaded}

\begin{verbatim}
##       w2   w5  bar
## w2  0.00 1.50 0.75
## w5  1.50 0.00 0.75
## bar 0.75 0.75 0.00
\end{verbatim}

\hypertarget{question-4-calculer-linertie-totale-de-lensemble-de-donnuxe9es}{%
\subsection{Question 4 : Calculer l'inertie totale de l'ensemble de
données}\label{question-4-calculer-linertie-totale-de-lensemble-de-donnuxe9es}}

\begin{Shaded}
\begin{Highlighting}[]
\NormalTok{dist_totale=}\KeywordTok{distance}\NormalTok{(data)}

\NormalTok{I_totale=}\DecValTok{1}\OperatorTok{/}\DecValTok{6}\OperatorTok{*}\NormalTok{(}\KeywordTok{sum}\NormalTok{(dist_totale[,}\StringTok{'bar'}\NormalTok{]}\OperatorTok{^}\DecValTok{2}\NormalTok{)) }\CommentTok{# sélectionner la colonne bar}
\NormalTok{I_totale}
\end{Highlighting}
\end{Shaded}

\begin{verbatim}
## [1] 8.756944
\end{verbatim}

\begin{Shaded}
\begin{Highlighting}[]
\CommentTok{# Méthode prof}
\NormalTok{n<-}\KeywordTok{nrow}\NormalTok{(data)}
\NormalTok{Inertie_total<-}\KeywordTok{sum}\NormalTok{((}\KeywordTok{distance}\NormalTok{(data)}\OperatorTok{^}\DecValTok{2}\NormalTok{)[n}\OperatorTok{+}\DecValTok{1}\NormalTok{,}\DecValTok{1}\OperatorTok{:}\NormalTok{n])}\OperatorTok{/}\NormalTok{n}
\NormalTok{Inertie_total}
\end{Highlighting}
\end{Shaded}

\begin{verbatim}
## [1] 8.756944
\end{verbatim}

\hypertarget{question-5-calculer-linertie-interclasse-intraclasse-et-le-pourcentage-dinertie-expliquuxe9}{%
\subsection{Question 5 : Calculer l'inertie interclasse, intraclasse et
le pourcentage d'inertie
expliqué}\label{question-5-calculer-linertie-interclasse-intraclasse-et-le-pourcentage-dinertie-expliquuxe9}}

\begin{Shaded}
\begin{Highlighting}[]
\NormalTok{bary_global<-}\KeywordTok{barycentre}\NormalTok{(data)}
\NormalTok{bary_global}
\end{Highlighting}
\end{Shaded}

\begin{verbatim}
##       X1       X2 
## 3.416667 4.166667
\end{verbatim}

\begin{Shaded}
\begin{Highlighting}[]
\NormalTok{I_inter<-}\ControlFlowTok{function}\NormalTok{(vect,bary)\{}
\NormalTok{  n<-}\KeywordTok{nrow}\NormalTok{(vect)}
\NormalTok{  vect_}\DecValTok{2}\NormalTok{<-}\KeywordTok{rbind}\NormalTok{(vect,bary)}
\NormalTok{  Inertie<-}\KeywordTok{sum}\NormalTok{((}\KeywordTok{distance}\NormalTok{(vect_}\DecValTok{2}\NormalTok{)}\OperatorTok{^}\DecValTok{2}\NormalTok{)[}\DecValTok{6}\OperatorTok{+}\DecValTok{1}\NormalTok{,}\DecValTok{1}\OperatorTok{:}\NormalTok{n])}\OperatorTok{/}\DecValTok{6}
\NormalTok{\}}
\end{Highlighting}
\end{Shaded}

\begin{Shaded}
\begin{Highlighting}[]
\CommentTok{# barycentre global}
\NormalTok{G <-}\KeywordTok{barycentre}\NormalTok{(data)}

\CommentTok{# Matrice des barycentres}
\NormalTok{bar <-}\StringTok{ }\KeywordTok{rbind}\NormalTok{(G1_}\DecValTok{3}\NormalTok{,G4,G2_}\DecValTok{5}\NormalTok{,G6,G)}
\NormalTok{nbar<-}\KeywordTok{nrow}\NormalTok{(bar)}

\CommentTok{#distance au carré entre les différents barycentres et le barycentre global}
\NormalTok{ecarts <-(}\KeywordTok{as.matrix}\NormalTok{(}\KeywordTok{dist}\NormalTok{(bar))}\OperatorTok{^}\DecValTok{2}\NormalTok{)[nbar,}\DecValTok{1}\OperatorTok{:}\NormalTok{nbar}\DecValTok{-1}\NormalTok{]}

\CommentTok{# effectifs de chaque classe}
\NormalTok{effectifs<-( }\KeywordTok{c}\NormalTok{(}\DecValTok{2}\NormalTok{,}\DecValTok{1}\NormalTok{,}\DecValTok{2}\NormalTok{,}\DecValTok{1}\NormalTok{))}

\NormalTok{Inertie_inter <-}\StringTok{ }\KeywordTok{sum}\NormalTok{(ecarts}\OperatorTok{*}\NormalTok{effectifs)}\OperatorTok{/}\KeywordTok{sum}\NormalTok{(effectifs)}

\NormalTok{temp}\FloatTok{.1}\NormalTok{ <-}\StringTok{ }\NormalTok{(d1}\OperatorTok{^}\DecValTok{2}\NormalTok{)[}\KeywordTok{nrow}\NormalTok{(d1),}\DecValTok{1}\OperatorTok{:}\KeywordTok{nrow}\NormalTok{(d1)}\OperatorTok{-}\DecValTok{1}\NormalTok{]}

\NormalTok{temp}\FloatTok{.2}\NormalTok{ <-}\StringTok{ }\NormalTok{(d2}\OperatorTok{^}\DecValTok{2}\NormalTok{)[}\KeywordTok{nrow}\NormalTok{(d2),}\DecValTok{1}\OperatorTok{:}\KeywordTok{nrow}\NormalTok{(d2)}\OperatorTok{-}\DecValTok{1}\NormalTok{]}

\NormalTok{vect.sum.dist<-}\KeywordTok{rbind}\NormalTok{(}\KeywordTok{mean}\NormalTok{(temp}\FloatTok{.1}\NormalTok{),}\DecValTok{0}\NormalTok{,}\KeywordTok{mean}\NormalTok{(temp}\FloatTok{.2}\NormalTok{),}\DecValTok{0}\NormalTok{)}

\NormalTok{Inertie_intra <-}\StringTok{ }\KeywordTok{sum}\NormalTok{(vect.sum.dist}\OperatorTok{*}\NormalTok{effectifs)}\OperatorTok{/}\KeywordTok{sum}\NormalTok{(effectifs)}
\NormalTok{Inertie_intra}
\end{Highlighting}
\end{Shaded}

\begin{verbatim}
## [1] 0.3541667
\end{verbatim}

\begin{Shaded}
\begin{Highlighting}[]
\NormalTok{Inertie_exp<-(}\DecValTok{1}\OperatorTok{-}\NormalTok{(Inertie_intra}\OperatorTok{/}\NormalTok{Inertie_total))}\OperatorTok{*}\DecValTok{100}
\NormalTok{Inertie_exp}
\end{Highlighting}
\end{Shaded}

\begin{verbatim}
## [1] 95.95559
\end{verbatim}

\hypertarget{question-6-repruxe9senter-graphiquement-le-nuage-de-point.}{%
\subsection{Question 6 : Représenter graphiquement le nuage de
point.}\label{question-6-repruxe9senter-graphiquement-le-nuage-de-point.}}

\begin{Shaded}
\begin{Highlighting}[]
\KeywordTok{plot}\NormalTok{(data)}
\end{Highlighting}
\end{Shaded}

\includegraphics{tp1_files/figure-latex/unnamed-chunk-13-1.pdf}

\hypertarget{quelle-segmentation-proposez-vous}{%
\subsubsection{Quelle segmentation proposez-vous
?}\label{quelle-segmentation-proposez-vous}}

On pourrait proposer la répartion suivante :
\{\{w1,w3\},\{w4,w6\},\{w2,w5\}\}

\hypertarget{calculer-le-pourcentage-dinertie-expliquuxe9e-par-cette-nouvelle-partition-et-comparer-uxe0-celle-de-la-question-2}{%
\subsubsection{Calculer le pourcentage d'inertie expliquée par cette
nouvelle partition et comparer à celle de la question
2}\label{calculer-le-pourcentage-dinertie-expliquuxe9e-par-cette-nouvelle-partition-et-comparer-uxe0-celle-de-la-question-2}}


\end{document}
