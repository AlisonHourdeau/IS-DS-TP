% Options for packages loaded elsewhere
\PassOptionsToPackage{unicode}{hyperref}
\PassOptionsToPackage{hyphens}{url}
%
\documentclass[
]{article}
\usepackage{lmodern}
\usepackage{amssymb,amsmath}
\usepackage{ifxetex,ifluatex}
\ifnum 0\ifxetex 1\fi\ifluatex 1\fi=0 % if pdftex
  \usepackage[T1]{fontenc}
  \usepackage[utf8]{inputenc}
  \usepackage{textcomp} % provide euro and other symbols
\else % if luatex or xetex
  \usepackage{unicode-math}
  \defaultfontfeatures{Scale=MatchLowercase}
  \defaultfontfeatures[\rmfamily]{Ligatures=TeX,Scale=1}
\fi
% Use upquote if available, for straight quotes in verbatim environments
\IfFileExists{upquote.sty}{\usepackage{upquote}}{}
\IfFileExists{microtype.sty}{% use microtype if available
  \usepackage[]{microtype}
  \UseMicrotypeSet[protrusion]{basicmath} % disable protrusion for tt fonts
}{}
\makeatletter
\@ifundefined{KOMAClassName}{% if non-KOMA class
  \IfFileExists{parskip.sty}{%
    \usepackage{parskip}
  }{% else
    \setlength{\parindent}{0pt}
    \setlength{\parskip}{6pt plus 2pt minus 1pt}}
}{% if KOMA class
  \KOMAoptions{parskip=half}}
\makeatother
\usepackage{xcolor}
\IfFileExists{xurl.sty}{\usepackage{xurl}}{} % add URL line breaks if available
\IfFileExists{bookmark.sty}{\usepackage{bookmark}}{\usepackage{hyperref}}
\hypersetup{
  pdftitle={Classification supervisée : fiche TP1, corrigé},
  pdfauthor={Vincent Vandewalle, Cristian Preda},
  hidelinks,
  pdfcreator={LaTeX via pandoc}}
\urlstyle{same} % disable monospaced font for URLs
\usepackage[margin=1in]{geometry}
\usepackage{color}
\usepackage{fancyvrb}
\newcommand{\VerbBar}{|}
\newcommand{\VERB}{\Verb[commandchars=\\\{\}]}
\DefineVerbatimEnvironment{Highlighting}{Verbatim}{commandchars=\\\{\}}
% Add ',fontsize=\small' for more characters per line
\usepackage{framed}
\definecolor{shadecolor}{RGB}{248,248,248}
\newenvironment{Shaded}{\begin{snugshade}}{\end{snugshade}}
\newcommand{\AlertTok}[1]{\textcolor[rgb]{0.94,0.16,0.16}{#1}}
\newcommand{\AnnotationTok}[1]{\textcolor[rgb]{0.56,0.35,0.01}{\textbf{\textit{#1}}}}
\newcommand{\AttributeTok}[1]{\textcolor[rgb]{0.77,0.63,0.00}{#1}}
\newcommand{\BaseNTok}[1]{\textcolor[rgb]{0.00,0.00,0.81}{#1}}
\newcommand{\BuiltInTok}[1]{#1}
\newcommand{\CharTok}[1]{\textcolor[rgb]{0.31,0.60,0.02}{#1}}
\newcommand{\CommentTok}[1]{\textcolor[rgb]{0.56,0.35,0.01}{\textit{#1}}}
\newcommand{\CommentVarTok}[1]{\textcolor[rgb]{0.56,0.35,0.01}{\textbf{\textit{#1}}}}
\newcommand{\ConstantTok}[1]{\textcolor[rgb]{0.00,0.00,0.00}{#1}}
\newcommand{\ControlFlowTok}[1]{\textcolor[rgb]{0.13,0.29,0.53}{\textbf{#1}}}
\newcommand{\DataTypeTok}[1]{\textcolor[rgb]{0.13,0.29,0.53}{#1}}
\newcommand{\DecValTok}[1]{\textcolor[rgb]{0.00,0.00,0.81}{#1}}
\newcommand{\DocumentationTok}[1]{\textcolor[rgb]{0.56,0.35,0.01}{\textbf{\textit{#1}}}}
\newcommand{\ErrorTok}[1]{\textcolor[rgb]{0.64,0.00,0.00}{\textbf{#1}}}
\newcommand{\ExtensionTok}[1]{#1}
\newcommand{\FloatTok}[1]{\textcolor[rgb]{0.00,0.00,0.81}{#1}}
\newcommand{\FunctionTok}[1]{\textcolor[rgb]{0.00,0.00,0.00}{#1}}
\newcommand{\ImportTok}[1]{#1}
\newcommand{\InformationTok}[1]{\textcolor[rgb]{0.56,0.35,0.01}{\textbf{\textit{#1}}}}
\newcommand{\KeywordTok}[1]{\textcolor[rgb]{0.13,0.29,0.53}{\textbf{#1}}}
\newcommand{\NormalTok}[1]{#1}
\newcommand{\OperatorTok}[1]{\textcolor[rgb]{0.81,0.36,0.00}{\textbf{#1}}}
\newcommand{\OtherTok}[1]{\textcolor[rgb]{0.56,0.35,0.01}{#1}}
\newcommand{\PreprocessorTok}[1]{\textcolor[rgb]{0.56,0.35,0.01}{\textit{#1}}}
\newcommand{\RegionMarkerTok}[1]{#1}
\newcommand{\SpecialCharTok}[1]{\textcolor[rgb]{0.00,0.00,0.00}{#1}}
\newcommand{\SpecialStringTok}[1]{\textcolor[rgb]{0.31,0.60,0.02}{#1}}
\newcommand{\StringTok}[1]{\textcolor[rgb]{0.31,0.60,0.02}{#1}}
\newcommand{\VariableTok}[1]{\textcolor[rgb]{0.00,0.00,0.00}{#1}}
\newcommand{\VerbatimStringTok}[1]{\textcolor[rgb]{0.31,0.60,0.02}{#1}}
\newcommand{\WarningTok}[1]{\textcolor[rgb]{0.56,0.35,0.01}{\textbf{\textit{#1}}}}
\usepackage{longtable,booktabs}
% Correct order of tables after \paragraph or \subparagraph
\usepackage{etoolbox}
\makeatletter
\patchcmd\longtable{\par}{\if@noskipsec\mbox{}\fi\par}{}{}
\makeatother
% Allow footnotes in longtable head/foot
\IfFileExists{footnotehyper.sty}{\usepackage{footnotehyper}}{\usepackage{footnote}}
\makesavenoteenv{longtable}
\usepackage{graphicx,grffile}
\makeatletter
\def\maxwidth{\ifdim\Gin@nat@width>\linewidth\linewidth\else\Gin@nat@width\fi}
\def\maxheight{\ifdim\Gin@nat@height>\textheight\textheight\else\Gin@nat@height\fi}
\makeatother
% Scale images if necessary, so that they will not overflow the page
% margins by default, and it is still possible to overwrite the defaults
% using explicit options in \includegraphics[width, height, ...]{}
\setkeys{Gin}{width=\maxwidth,height=\maxheight,keepaspectratio}
% Set default figure placement to htbp
\makeatletter
\def\fps@figure{htbp}
\makeatother
\setlength{\emergencystretch}{3em} % prevent overfull lines
\providecommand{\tightlist}{%
  \setlength{\itemsep}{0pt}\setlength{\parskip}{0pt}}
\setcounter{secnumdepth}{5}

\title{Classification supervisée : fiche TP1, corrigé}
\author{Vincent Vandewalle, Cristian Preda}
\date{}

\begin{document}
\maketitle

{
\setcounter{tocdepth}{2}
\tableofcontents
}
\begin{Shaded}
\begin{Highlighting}[]
\NormalTok{knitr}\OperatorTok{::}\NormalTok{opts_chunk}\OperatorTok{$}\KeywordTok{set}\NormalTok{(}\DataTypeTok{echo =} \OtherTok{TRUE}\NormalTok{, }\DataTypeTok{error =} \OtherTok{TRUE}\NormalTok{)}
\end{Highlighting}
\end{Shaded}

\hypertarget{reprise-des-exemples-du-cours}{%
\section{Reprise des exemples du
cours}\label{reprise-des-exemples-du-cours}}

Lancer les commandes suivantes pour retrouver les résultats des
exercices de la première séance de cours.

\hypertarget{test-du-chi-deux}{%
\subsection{Test du Chi-deux}\label{test-du-chi-deux}}

\begin{Shaded}
\begin{Highlighting}[]
\NormalTok{V3V1<-}\KeywordTok{matrix}\NormalTok{(}\KeywordTok{c}\NormalTok{(}\DecValTok{30}\NormalTok{,}\DecValTok{20}\NormalTok{,}\DecValTok{30}\NormalTok{,}\DecValTok{20}\NormalTok{,}\DecValTok{10}\NormalTok{,}\DecValTok{15}\NormalTok{,}\DecValTok{10}\NormalTok{,}\DecValTok{15}\NormalTok{),}\DecValTok{4}\NormalTok{,}\DecValTok{2}\NormalTok{,}\DataTypeTok{byrow=}\OtherTok{TRUE}\NormalTok{)}
\NormalTok{V3V1}
\end{Highlighting}
\end{Shaded}

\begin{verbatim}
##      [,1] [,2]
## [1,]   30   20
## [2,]   30   20
## [3,]   10   15
## [4,]   10   15
\end{verbatim}

\begin{Shaded}
\begin{Highlighting}[]
\NormalTok{chi2 =}\StringTok{ }\KeywordTok{chisq.test}\NormalTok{(V3V1)}
\KeywordTok{str}\NormalTok{(chi2)}
\end{Highlighting}
\end{Shaded}

\begin{verbatim}
## List of 9
##  $ statistic: Named num 5.36
##   ..- attr(*, "names")= chr "X-squared"
##  $ parameter: Named int 3
##   ..- attr(*, "names")= chr "df"
##  $ p.value  : num 0.147
##  $ method   : chr "Pearson's Chi-squared test"
##  $ data.name: chr "V3V1"
##  $ observed : num [1:4, 1:2] 30 30 10 10 20 20 15 15
##  $ expected : num [1:4, 1:2] 26.7 26.7 13.3 13.3 23.3 ...
##  $ residuals: num [1:4, 1:2] 0.645 0.645 -0.913 -0.913 -0.69 ...
##  $ stdres   : num [1:4, 1:2] 1.16 1.16 -1.46 -1.46 -1.16 ...
##  - attr(*, "class")= chr "htest"
\end{verbatim}

\begin{Shaded}
\begin{Highlighting}[]
\NormalTok{chi2}
\end{Highlighting}
\end{Shaded}

\begin{verbatim}
## 
##  Pearson's Chi-squared test
## 
## data:  V3V1
## X-squared = 5.3571, df = 3, p-value = 0.1474
\end{verbatim}

\begin{Shaded}
\begin{Highlighting}[]
\DecValTok{1}\OperatorTok{-}\KeywordTok{pchisq}\NormalTok{(}\FloatTok{5.3571}\NormalTok{,}\DecValTok{3}\NormalTok{)}
\end{Highlighting}
\end{Shaded}

\begin{verbatim}
## [1] 0.1474399
\end{verbatim}

\begin{Shaded}
\begin{Highlighting}[]
\KeywordTok{sum}\NormalTok{(chi2}\OperatorTok{$}\NormalTok{residuals}\OperatorTok{^}\DecValTok{2}\NormalTok{)}
\end{Highlighting}
\end{Shaded}

\begin{verbatim}
## [1] 5.357143
\end{verbatim}

\begin{enumerate}
\def\labelenumi{\arabic{enumi}.}
\tightlist
\item
  Commenter code et résultats : ici on réalise un test du Chi-deux
  d'indépendance, on ne rejetterai pas l'hypothèse d'indépendance au
  risque \(\alpha = 0,05\).
\end{enumerate}

\hypertarget{test-de-fisher}{%
\subsection{Test de Fisher}\label{test-de-fisher}}

\begin{Shaded}
\begin{Highlighting}[]
\NormalTok{x <-}\StringTok{ }\KeywordTok{c}\NormalTok{(}\DecValTok{4}\NormalTok{,}\DecValTok{5}\NormalTok{,}\DecValTok{7}\NormalTok{,}\DecValTok{8}\NormalTok{,}\DecValTok{9}\NormalTok{,}\DecValTok{2}\NormalTok{,}\DecValTok{3}\NormalTok{,}\DecValTok{4}\NormalTok{,}\DecValTok{6}\NormalTok{,}\DecValTok{7}\NormalTok{,}\DecValTok{8}\NormalTok{)}
\NormalTok{y <-}\StringTok{ }\KeywordTok{c}\NormalTok{(}\KeywordTok{rep}\NormalTok{(}\DecValTok{0}\NormalTok{,}\DecValTok{5}\NormalTok{),}\KeywordTok{rep}\NormalTok{(}\DecValTok{1}\NormalTok{,}\DecValTok{6}\NormalTok{))}
\KeywordTok{cbind}\NormalTok{(x,y)}
\end{Highlighting}
\end{Shaded}

\begin{verbatim}
##       x y
##  [1,] 4 0
##  [2,] 5 0
##  [3,] 7 0
##  [4,] 8 0
##  [5,] 9 0
##  [6,] 2 1
##  [7,] 3 1
##  [8,] 4 1
##  [9,] 6 1
## [10,] 7 1
## [11,] 8 1
\end{verbatim}

\begin{Shaded}
\begin{Highlighting}[]
\KeywordTok{lm}\NormalTok{(x}\OperatorTok{~}\KeywordTok{factor}\NormalTok{(y))}
\end{Highlighting}
\end{Shaded}

\begin{verbatim}
## 
## Call:
## lm(formula = x ~ factor(y))
## 
## Coefficients:
## (Intercept)   factor(y)1  
##         6.6         -1.6
\end{verbatim}

\begin{Shaded}
\begin{Highlighting}[]
\KeywordTok{factor}\NormalTok{(y)}
\end{Highlighting}
\end{Shaded}

\begin{verbatim}
##  [1] 0 0 0 0 0 1 1 1 1 1 1
## Levels: 0 1
\end{verbatim}

\begin{Shaded}
\begin{Highlighting}[]
\KeywordTok{anova}\NormalTok{(}\KeywordTok{lm}\NormalTok{(x}\OperatorTok{~}\NormalTok{y))}
\end{Highlighting}
\end{Shaded}

\begin{verbatim}
## Analysis of Variance Table
## 
## Response: x
##           Df Sum Sq Mean Sq F value Pr(>F)
## y          1  6.982  6.9818  1.3902 0.2686
## Residuals  9 45.200  5.0222
\end{verbatim}

\begin{Shaded}
\begin{Highlighting}[]
\NormalTok{SCF <-}\StringTok{ }\NormalTok{(}\KeywordTok{mean}\NormalTok{(x[}\DecValTok{1}\OperatorTok{:}\DecValTok{5}\NormalTok{])}\OperatorTok{-}\KeywordTok{mean}\NormalTok{(x))}\OperatorTok{^}\DecValTok{2}\OperatorTok{*}\DecValTok{5}\OperatorTok{+}\NormalTok{(}\KeywordTok{mean}\NormalTok{(x[}\DecValTok{6}\OperatorTok{:}\DecValTok{11}\NormalTok{])}\OperatorTok{-}\KeywordTok{mean}\NormalTok{(x))}\OperatorTok{^}\DecValTok{2}\OperatorTok{*}\DecValTok{6}
\NormalTok{SCR <-}\StringTok{ }\KeywordTok{sum}\NormalTok{(}\KeywordTok{c}\NormalTok{((x[}\DecValTok{1}\OperatorTok{:}\DecValTok{5}\NormalTok{]}\OperatorTok{-}\KeywordTok{mean}\NormalTok{(x[}\DecValTok{1}\OperatorTok{:}\DecValTok{5}\NormalTok{]))}\OperatorTok{^}\DecValTok{2}\NormalTok{,(x[}\DecValTok{6}\OperatorTok{:}\DecValTok{11}\NormalTok{]}\OperatorTok{-}\KeywordTok{mean}\NormalTok{(x[}\DecValTok{6}\OperatorTok{:}\DecValTok{11}\NormalTok{]))}\OperatorTok{^}\DecValTok{2}\NormalTok{))}
\NormalTok{Fstat <-}\StringTok{ }\NormalTok{(SCF}\OperatorTok{/}\DecValTok{1}\NormalTok{)}\OperatorTok{/}\NormalTok{(SCR}\OperatorTok{/}\DecValTok{9}\NormalTok{)}
\NormalTok{pval <-}\StringTok{ }\KeywordTok{pf}\NormalTok{(Fstat,}\DecValTok{1}\NormalTok{,}\DecValTok{9}\NormalTok{,}\DataTypeTok{lower.tail=}\OtherTok{FALSE}\NormalTok{)}
\NormalTok{pval}
\end{Highlighting}
\end{Shaded}

\begin{verbatim}
## [1] 0.2686035
\end{verbatim}

\begin{Shaded}
\begin{Highlighting}[]
\NormalTok{Rsq <-}\StringTok{ }\NormalTok{SCF}\OperatorTok{/}\NormalTok{(SCF}\OperatorTok{+}\NormalTok{SCR)}
\NormalTok{Rsq}
\end{Highlighting}
\end{Shaded}

\begin{verbatim}
## [1] 0.1337979
\end{verbatim}

\begin{Shaded}
\begin{Highlighting}[]
\KeywordTok{c}\NormalTok{(SCF,SCR,Fstat,pval,Rsq)}
\end{Highlighting}
\end{Shaded}

\begin{verbatim}
## [1]  6.9818182 45.2000000  1.3901850  0.2686035  0.1337979
\end{verbatim}

\begin{Shaded}
\begin{Highlighting}[]
\KeywordTok{summary}\NormalTok{(}\KeywordTok{lm}\NormalTok{(x}\OperatorTok{~}\NormalTok{y))}\OperatorTok{$}\NormalTok{r.squared}
\end{Highlighting}
\end{Shaded}

\begin{verbatim}
## [1] 0.1337979
\end{verbatim}

\begin{Shaded}
\begin{Highlighting}[]
\KeywordTok{pchisq}\NormalTok{(}\FloatTok{6.585}\OperatorTok{^}\DecValTok{2}\NormalTok{,}\DataTypeTok{df =} \DecValTok{1}\NormalTok{, }\DataTypeTok{lower.tail =} \OtherTok{FALSE}\NormalTok{)}
\end{Highlighting}
\end{Shaded}

\begin{verbatim}
## [1] 4.548854e-11
\end{verbatim}

\begin{enumerate}
\def\labelenumi{\arabic{enumi}.}
\setcounter{enumi}{1}
\tightlist
\item
  Commenter code et résultats : Ici on a réalisé un test de l'ANOVA,
  sous \(H_0\) la statistique de test suit une loi de Fisher a \(K-1\),
  \(n-K\) degré de liberté, avec \(K\) le nombre de classes et \(n\) le
  nombre de données. Ici la probabilité critique est de \(0,2686\) donc
  la variable \(y\) n'a pas d'effet significatif sur la variable \(x\)
  au risque \(\alpha = 0,05\).
\end{enumerate}

\textbf{Attention} : en classification supervisée on veut prédire \(y\)
à partir de \(x\) ! Mais le fait que la variance de \(x\) soit bien
expliquée par \(y\) nous donne un bon indicateur du pouvoir prédictif
\(x\) sur \(y\).

\hypertarget{analyse-pruxe9liminaire-du-jeu-de-donnuxe9es-iris-anova-et-manova}{%
\section{\texorpdfstring{Analyse préliminaire du jeu de données
\texttt{iris}, ANOVA et
MANOVA}{Analyse préliminaire du jeu de données iris, ANOVA et MANOVA}}\label{analyse-pruxe9liminaire-du-jeu-de-donnuxe9es-iris-anova-et-manova}}

\hypertarget{analyse-pruxe9liminaire-du-jeu-de-donnuxe9es-iris}{%
\subsection{\texorpdfstring{Analyse préliminaire du jeu de données
\texttt{iris}}{Analyse préliminaire du jeu de données iris}}\label{analyse-pruxe9liminaire-du-jeu-de-donnuxe9es-iris}}

Dans cette partie on utilisera les données \texttt{iris}.

\begin{enumerate}
\def\labelenumi{\arabic{enumi}.}
\setcounter{enumi}{2}
\tightlist
\item
  Faire \texttt{data("iris")} dans R.
\end{enumerate}

\begin{Shaded}
\begin{Highlighting}[]
\KeywordTok{data}\NormalTok{(}\StringTok{"iris"}\NormalTok{)}
\KeywordTok{head}\NormalTok{(iris)}
\end{Highlighting}
\end{Shaded}

\begin{verbatim}
##   Sepal.Length Sepal.Width Petal.Length Petal.Width Species
## 1          5.1         3.5          1.4         0.2  setosa
## 2          4.9         3.0          1.4         0.2  setosa
## 3          4.7         3.2          1.3         0.2  setosa
## 4          4.6         3.1          1.5         0.2  setosa
## 5          5.0         3.6          1.4         0.2  setosa
## 6          5.4         3.9          1.7         0.4  setosa
\end{verbatim}

\begin{enumerate}
\def\labelenumi{\arabic{enumi}.}
\setcounter{enumi}{3}
\tightlist
\item
  Renommer les variables ``Sepal.Length'', ``Sepal.Width'',
  ``Petal.Length'', ``Petal.Width'',``Species'' en ``X1'', ``X2'',
  ``X3'', ``X4'', ``Y''.
\end{enumerate}

\begin{Shaded}
\begin{Highlighting}[]
\KeywordTok{names}\NormalTok{(iris) <-}\StringTok{ }\KeywordTok{c}\NormalTok{(}\StringTok{"X1"}\NormalTok{,}\StringTok{"X2"}\NormalTok{,}\StringTok{"X3"}\NormalTok{,}\StringTok{"X4"}\NormalTok{,}\StringTok{"Y"}\NormalTok{) }
\end{Highlighting}
\end{Shaded}

\begin{enumerate}
\def\labelenumi{\arabic{enumi}.}
\setcounter{enumi}{4}
\tightlist
\item
  Représenter graphiquement le lien entre X1 et Y :
\end{enumerate}

\begin{Shaded}
\begin{Highlighting}[]
\KeywordTok{library}\NormalTok{(dplyr)}
\end{Highlighting}
\end{Shaded}

\begin{verbatim}
## 
## Attaching package: 'dplyr'
\end{verbatim}

\begin{verbatim}
## The following objects are masked from 'package:stats':
## 
##     filter, lag
\end{verbatim}

\begin{verbatim}
## The following objects are masked from 'package:base':
## 
##     intersect, setdiff, setequal, union
\end{verbatim}

\begin{Shaded}
\begin{Highlighting}[]
\KeywordTok{library}\NormalTok{(ggplot2)}
\NormalTok{iris }\OperatorTok\StringTok{ }
\StringTok{  }\KeywordTok{ggplot}\NormalTok{(}\KeywordTok{aes}\NormalTok{(}\DataTypeTok{x =}\NormalTok{ Y, }\DataTypeTok{y =}\NormalTok{ X1)) }\OperatorTok{+}\StringTok{ }
\StringTok{  }\KeywordTok{geom_boxplot}\NormalTok{()}
\end{Highlighting}
\end{Shaded}

\includegraphics{corrige_TP1_files/figure-latex/boxplot X1 sachant Y-1.pdf}

Puis faire de même pour les autres variables :

\begin{Shaded}
\begin{Highlighting}[]
\KeywordTok{library}\NormalTok{(}\StringTok{"tidyr"}\NormalTok{)}
\NormalTok{iris }\OperatorTok\StringTok{ }
\StringTok{  }\KeywordTok{gather}\NormalTok{(}\StringTok{"variable"}\NormalTok{,}\StringTok{"mesure"}\NormalTok{,}\OperatorTok{-}\NormalTok{Y) }\OperatorTok\StringTok{ }
\StringTok{  }\KeywordTok{ggplot}\NormalTok{(}\KeywordTok{aes}\NormalTok{(}\DataTypeTok{x =}\NormalTok{ Y, }\DataTypeTok{y =}\NormalTok{ mesure)) }\OperatorTok{+}\StringTok{ }
\StringTok{  }\KeywordTok{geom_boxplot}\NormalTok{() }\OperatorTok{+}
\StringTok{  }\KeywordTok{facet_wrap}\NormalTok{(}\OperatorTok{~}\StringTok{ }\NormalTok{variable, }\DataTypeTok{scales =} \StringTok{"free_y"}\NormalTok{)}
\end{Highlighting}
\end{Shaded}

\includegraphics{corrige_TP1_files/figure-latex/boxplot des X en fonction de Y-1.pdf}

Commenter : On voit que les variable pour lesquelles les classes sont
les mieux séparées sont les variable \(X_3\) et \(X_4\).

\hypertarget{anova}{%
\subsection{ANOVA}\label{anova}}

Réaliser l'ANOVA de X1 en fonction de Y et obtenir le \(R^2\) associé,
faire de même pour les autres variables. A partir des p-values, indiquer
si la variable Y a une influence sur l'ensemble des variables. Quelle
est la variable la mieux expliquée par Y ?

Ajustement du modèle linéaire pour X1

\begin{Shaded}
\begin{Highlighting}[]
\KeywordTok{lm}\NormalTok{(X1 }\OperatorTok{~}\StringTok{ }\NormalTok{Y, }\DataTypeTok{data =}\NormalTok{ iris)}
\end{Highlighting}
\end{Shaded}

\begin{verbatim}
## 
## Call:
## lm(formula = X1 ~ Y, data = iris)
## 
## Coefficients:
## (Intercept)  Yversicolor   Yvirginica  
##       5.006        0.930        1.582
\end{verbatim}

\begin{Shaded}
\begin{Highlighting}[]
\KeywordTok{shapiro.test}\NormalTok{(iris}\OperatorTok{$}\NormalTok{X1[iris}\OperatorTok{$}\NormalTok{Y }\OperatorTok{==}\StringTok{ "setosa"}\NormalTok{])}
\end{Highlighting}
\end{Shaded}

\begin{verbatim}
## 
##  Shapiro-Wilk normality test
## 
## data:  iris$X1[iris$Y == "setosa"]
## W = 0.9777, p-value = 0.4595
\end{verbatim}

\begin{Shaded}
\begin{Highlighting}[]
\CommentTok{# Test de normalité groupe par groupe : }
\KeywordTok{by}\NormalTok{(}\DataTypeTok{data =}\NormalTok{ iris}\OperatorTok{$}\NormalTok{X1, }\DataTypeTok{INDICES =}\NormalTok{ iris}\OperatorTok{$}\NormalTok{Y, shapiro.test)}
\end{Highlighting}
\end{Shaded}

\begin{verbatim}
## iris$Y: setosa
## 
##  Shapiro-Wilk normality test
## 
## data:  dd[x, ]
## W = 0.9777, p-value = 0.4595
## 
## ------------------------------------------------------------ 
## iris$Y: versicolor
## 
##  Shapiro-Wilk normality test
## 
## data:  dd[x, ]
## W = 0.97784, p-value = 0.4647
## 
## ------------------------------------------------------------ 
## iris$Y: virginica
## 
##  Shapiro-Wilk normality test
## 
## data:  dd[x, ]
## W = 0.97118, p-value = 0.2583
\end{verbatim}

\begin{Shaded}
\begin{Highlighting}[]
\CommentTok{# Ici accepte l'hypothèse de normalité dans chacune des classes}

\CommentTok{# Test d'homogénité des variances : }
\KeywordTok{bartlett.test}\NormalTok{(iris}\OperatorTok{$}\NormalTok{X1, iris}\OperatorTok{$}\NormalTok{Y)}
\end{Highlighting}
\end{Shaded}

\begin{verbatim}
## 
##  Bartlett test of homogeneity of variances
## 
## data:  iris$X1 and iris$Y
## Bartlett's K-squared = 16.006, df = 2, p-value = 0.0003345
\end{verbatim}

\begin{Shaded}
\begin{Highlighting}[]
\CommentTok{# p-value = 0.0003345}
\CommentTok{# On rejette l'homogénéité des variances}
\CommentTok{# Attention : conclusion du test de l'ANOVA possiblement erronnées}

\KeywordTok{summary}\NormalTok{(}\KeywordTok{lm}\NormalTok{(X1 }\OperatorTok{~}\StringTok{ }\NormalTok{Y, }\DataTypeTok{data =}\NormalTok{ iris)) }\CommentTok{# Résumé}
\end{Highlighting}
\end{Shaded}

\begin{verbatim}
## 
## Call:
## lm(formula = X1 ~ Y, data = iris)
## 
## Residuals:
##     Min      1Q  Median      3Q     Max 
## -1.6880 -0.3285 -0.0060  0.3120  1.3120 
## 
## Coefficients:
##             Estimate Std. Error t value Pr(>|t|)    
## (Intercept)   5.0060     0.0728  68.762  < 2e-16 ***
## Yversicolor   0.9300     0.1030   9.033 8.77e-16 ***
## Yvirginica    1.5820     0.1030  15.366  < 2e-16 ***
## ---
## Signif. codes:  0 '***' 0.001 '**' 0.01 '*' 0.05 '.' 0.1 ' ' 1
## 
## Residual standard error: 0.5148 on 147 degrees of freedom
## Multiple R-squared:  0.6187, Adjusted R-squared:  0.6135 
## F-statistic: 119.3 on 2 and 147 DF,  p-value: < 2.2e-16
\end{verbatim}

\begin{Shaded}
\begin{Highlighting}[]
\KeywordTok{summary}\NormalTok{(}\KeywordTok{lm}\NormalTok{(X1 }\OperatorTok{~}\StringTok{ }\NormalTok{Y, }\DataTypeTok{data =}\NormalTok{ iris))}\OperatorTok{$}\NormalTok{r.squared}
\end{Highlighting}
\end{Shaded}

\begin{verbatim}
## [1] 0.6187057
\end{verbatim}

Extension à chacune des variables

\begin{Shaded}
\begin{Highlighting}[]
\KeywordTok{sapply}\NormalTok{(}\KeywordTok{names}\NormalTok{(iris)[}\OperatorTok{-}\DecValTok{5}\NormalTok{], }
       \ControlFlowTok{function}\NormalTok{(x) }\KeywordTok{summary}\NormalTok{(}\KeywordTok{lm}\NormalTok{(}\KeywordTok{as.formula}\NormalTok{(}\KeywordTok{paste}\NormalTok{(x,}\StringTok{"~ Y"}\NormalTok{)), }
                              \DataTypeTok{data =}\NormalTok{ iris))}\OperatorTok{$}\NormalTok{r.squared)}
\end{Highlighting}
\end{Shaded}

\begin{verbatim}
##        X1        X2        X3        X4 
## 0.6187057 0.4007828 0.9413717 0.9288829
\end{verbatim}

\begin{enumerate}
\def\labelenumi{\arabic{enumi}.}
\setcounter{enumi}{5}
\tightlist
\item
  Commenter ces résultats : Ici la variable que dont la variance est la
  mieux expliquée par \(Y\) est la variable \(X_4\)
\end{enumerate}

\[
R^2_{X_4 / Y} = \frac{\mbox{Variance de }X_4 \mbox{ expliquée par } Y }{\mbox{Variance de }X_4} = 92,8\%
\]

Calcul de l'ANOVA (calcul de la p-value du test)

\begin{Shaded}
\begin{Highlighting}[]
\KeywordTok{anova}\NormalTok{(}\KeywordTok{lm}\NormalTok{(X1}\OperatorTok{~}\NormalTok{Y,}\DataTypeTok{data=}\NormalTok{iris)) }
\end{Highlighting}
\end{Shaded}

\begin{verbatim}
## Analysis of Variance Table
## 
## Response: X1
##            Df Sum Sq Mean Sq F value    Pr(>F)    
## Y           2 63.212  31.606  119.26 < 2.2e-16 ***
## Residuals 147 38.956   0.265                      
## ---
## Signif. codes:  0 '***' 0.001 '**' 0.01 '*' 0.05 '.' 0.1 ' ' 1
\end{verbatim}

\begin{Shaded}
\begin{Highlighting}[]
\KeywordTok{anova}\NormalTok{(}\KeywordTok{lm}\NormalTok{(X1}\OperatorTok{~}\NormalTok{Y,}\DataTypeTok{data=}\NormalTok{iris))}\OperatorTok{$}\StringTok{`}\DataTypeTok{Pr(>F)}\StringTok{`}
\end{Highlighting}
\end{Shaded}

\begin{verbatim}
## [1] 1.669669e-31           NA
\end{verbatim}

\begin{Shaded}
\begin{Highlighting}[]
\KeywordTok{anova}\NormalTok{(}\KeywordTok{lm}\NormalTok{(X1}\OperatorTok{~}\NormalTok{Y,}\DataTypeTok{data=}\NormalTok{iris))}\OperatorTok{$}\StringTok{`}\DataTypeTok{Pr(>F)}\StringTok{`}\NormalTok{[}\DecValTok{1}\NormalTok{]}
\end{Highlighting}
\end{Shaded}

\begin{verbatim}
## [1] 1.669669e-31
\end{verbatim}

Extension à chacune des variables

\begin{Shaded}
\begin{Highlighting}[]
\KeywordTok{sapply}\NormalTok{(}\KeywordTok{names}\NormalTok{(iris)[}\OperatorTok{-}\DecValTok{5}\NormalTok{], }
       \ControlFlowTok{function}\NormalTok{(x) }\KeywordTok{anova}\NormalTok{(}\KeywordTok{lm}\NormalTok{(}\KeywordTok{as.formula}\NormalTok{(}\KeywordTok{paste}\NormalTok{(x,}\StringTok{"~ Y"}\NormalTok{)), }
                            \DataTypeTok{data =}\NormalTok{ iris))}\OperatorTok{$}\StringTok{`}\DataTypeTok{Pr(>F)}\StringTok{`}\NormalTok{[}\DecValTok{1}\NormalTok{])}
\end{Highlighting}
\end{Shaded}

\begin{verbatim}
##           X1           X2           X3           X4 
## 1.669669e-31 4.492017e-17 2.856777e-91 4.169446e-85
\end{verbatim}

\begin{enumerate}
\def\labelenumi{\arabic{enumi}.}
\setcounter{enumi}{6}
\tightlist
\item
  Commentez ces résultats : la sous-espèce a-t'elle un effet
  significatif sur l'espérance de X1 ? de X2 ? de X3 ? de X4 ? Sur
  l'espérance de
  \(X = \begin{pmatrix} X_1 \\ X_2 \\ X_3 \\ X_4 \end{pmatrix}\) ?
\end{enumerate}

Oui la sous-espéce à un effet significatif sur chacune des variables
explicatives.

Ici vous avez testé : \[
H_{0j} = \{ \mu_{1j} = \mu_{2j} = \mu_{3j} \} \mbox{ contre } H_{1j} = \{\exists i \neq i'|  \mu_{ij} \neq \mu_{i'j} \},
\] pour \(j \in \{1,2,3,4\}\), c'est-à-dire pour chacune des variables
séparément.

Mais, ce que nous souhaitons tester ici est : y-a-t'il une différence
entre groupes pour au moins une des variables ? : \[
H_{0} = \{ \mu_{1} = \mu_{2} = \mu_{3} \} \mbox{ contre } H_{1} = \{\exists j \in \{1,2,3,4\}, \exists i \neq i'|  \mu_{ij} \neq \mu_{i'j} \}
\] avec
\(\mu_i = \begin{pmatrix} \mu_{i1} \\ \mu_{i2} \\ \mu_{i3} \\ \mu_{i4}\end{pmatrix}\).

Comment recoller les morceaux ???

Remarquons d'abord que : \[
H_0 = \cap_{j=1}^{4} H_{0j} = H_{01} \cap H_{02} \cap H_{03} \cap H_{04}
\]

Ainsi \(H_0\) est fausse du moment qu'au moins une des \(H_{0j}\) est
fausse. La question est alors quel est le risque de première espèce
\(\alpha_{global}\) de rejeter \(H_0\) à tord quand on se donne un
risque de première espèce \(\alpha\) de rejeter \(H_{0j}\) à tord pour
\(j \in \{1,2,3,4\}\) ? Et comment choisir \(\alpha\) de manière à
maintenir un risque global \(\alpha_{global}\) ?

On note \(p_j\) les probabilités critiques associées à chacune de
\(H_{0j}\). Sous \(H_{0j}\) on sait que \(p_j\) suit une loi uniforme
sur \([0;1]\) (\(p_j \sim U([0;1])\)). En notant \(A_j\) l'événement
\(H_{0j}\) est rejeté, \(A_j = \{p_j \leq \alpha \}\).

\[
P_{H_0}(\mbox{rejet de } H_0 \mbox{ à tord}) =  P_{H_0}(\cup_{j=1}^{d} A_j) = P_{H_0}(\cup_{j=1}^{d}\{p_j \leq \alpha\}) 
\leq \sum_{j = 1}^{d}P_{H_{0j}}\left(p_j \leq \alpha\right) = d \times \alpha
\]

Ainsi, si on veut s'assurer que
\(P_{H_0}(\mbox{rejet de } H_0 \mbox{ à tord}) \leq \alpha_{global}\),
on peut choisir \(\alpha = \frac{\alpha_{global}}{d}\). Il s'agit de la
correction de \textbf{Bonferroni} (cette correction est plutôt frustre
et on peu parfois lui préférer d'autres corrections comme l'utilisation
du False Discovery Rate (FDR) qui vise à controler le pourcentage de
faux positifs).

\begin{enumerate}
\def\labelenumi{\arabic{enumi}.}
\setcounter{enumi}{7}
\tightlist
\item
  On se donne un risque de première espèce \(\alpha_{global} = 0,05\),
  réaliser l'ajustement de Bonferroni. Rejettez-vous \(H_0\) ?
\end{enumerate}

\begin{Shaded}
\begin{Highlighting}[]
\NormalTok{alpha_glo =}\StringTok{ }\FloatTok{0.05}
\NormalTok{d =}\StringTok{ }\DecValTok{4}
\NormalTok{alpha =}\StringTok{ }\NormalTok{alpha_glo}\OperatorTok{/}\NormalTok{d}
\NormalTok{alpha}
\end{Highlighting}
\end{Shaded}

\begin{verbatim}
## [1] 0.0125
\end{verbatim}

\begin{Shaded}
\begin{Highlighting}[]
\NormalTok{pvalue =}\StringTok{ }\KeywordTok{sapply}\NormalTok{(}\KeywordTok{names}\NormalTok{(iris)[}\OperatorTok{-}\DecValTok{5}\NormalTok{], }
       \ControlFlowTok{function}\NormalTok{(x) }\KeywordTok{anova}\NormalTok{(}\KeywordTok{lm}\NormalTok{(}\KeywordTok{as.formula}\NormalTok{(}\KeywordTok{paste}\NormalTok{(x,}\StringTok{"~ Y"}\NormalTok{)), }
                            \DataTypeTok{data =}\NormalTok{ iris))}\OperatorTok{$}\StringTok{`}\DataTypeTok{Pr(>F)}\StringTok{`}\NormalTok{[}\DecValTok{1}\NormalTok{])}
\NormalTok{pvalue}
\end{Highlighting}
\end{Shaded}

\begin{verbatim}
##           X1           X2           X3           X4 
## 1.669669e-31 4.492017e-17 2.856777e-91 4.169446e-85
\end{verbatim}

\begin{Shaded}
\begin{Highlighting}[]
\KeywordTok{any}\NormalTok{(pvalue }\OperatorTok{<}\StringTok{ }\NormalTok{alpha) }\CommentTok{# TRUE : moins une des }
\end{Highlighting}
\end{Shaded}

\begin{verbatim}
## [1] TRUE
\end{verbatim}

\begin{Shaded}
\begin{Highlighting}[]
\CommentTok{# p-valeurs est inférieure à 0,0125 donc on rejette H_0 au risque global alpha = 0,05 ! La distribution de X varie en fonction du groupe Y.}
\end{Highlighting}
\end{Shaded}

\hypertarget{manova}{%
\subsection{MANOVA}\label{manova}}

Contrairement à la situation précédente on souhaite tester directement
\(H_0\) contre \(H_1\), ce qui impose un modèle sur la distribution du
vecteur \(X\) sachant la classe \(Y\) :

\begin{itemize}
\tightlist
\item
  \(X | Y = k \sim \mathcal{N}_d(\mu_i;\Sigma_i)\) : Hypothèse de
  normalité sachant la classe
\item
  \(\Sigma_1 = \Sigma_2 = \cdots = \Sigma_K = \Sigma\) : Hypothèse
  d'homogénéité des variances
\end{itemize}

En utilisant la fonction \texttt{ggpairs} du package \texttt{GGally} on
représente les corrélations deux à deux entre les différentes variables
en fonction de la variable \texttt{Y} comme suit :

\begin{Shaded}
\begin{Highlighting}[]
\KeywordTok{library}\NormalTok{(GGally)}
\KeywordTok{ggpairs}\NormalTok{(iris, }\DataTypeTok{columns =} \DecValTok{1}\OperatorTok{:}\DecValTok{4}\NormalTok{, }\KeywordTok{aes}\NormalTok{(}\DataTypeTok{color =}\NormalTok{ Y, }\DataTypeTok{alpha =} \FloatTok{0.8}\NormalTok{))}
\end{Highlighting}
\end{Shaded}

\includegraphics{corrige_TP1_files/figure-latex/Nuages de point deux a deux-1.pdf}

\begin{enumerate}
\def\labelenumi{\arabic{enumi}.}
\setcounter{enumi}{8}
\tightlist
\item
  Commenter le graphique obtenus, que dire des hypothèses de normalité
  et d'homogénéité des variances ?
\end{enumerate}

Ici à l'allure du nuage de point on peut éventuellement admettre un
normalité classe par classe, cependant l'hypothèse d'homogénéité des
variance ne semble pas vérifiée (allure du nuage de point différente
d'une classe à l'autre).

A l'aide de la fonction \texttt{mshapiro.test} de la librairie
\texttt{mvnormtest} réaliser un test de normalité pour chacune des
classes :

\begin{Shaded}
\begin{Highlighting}[]
\CommentTok{# install.packages("mvnormtest")}
\KeywordTok{library}\NormalTok{(mvnormtest)}
\KeywordTok{mshapiro.test}\NormalTok{(}\KeywordTok{as.matrix}\NormalTok{(}\KeywordTok{t}\NormalTok{(iris[iris}\OperatorTok{$}\NormalTok{Y}\OperatorTok{==}\StringTok{"versicolor"}\NormalTok{,}\DecValTok{1}\OperatorTok{:}\DecValTok{4}\NormalTok{])))}
\end{Highlighting}
\end{Shaded}

\begin{verbatim}
## 
##  Shapiro-Wilk normality test
## 
## data:  Z
## W = 0.93043, p-value = 0.005739
\end{verbatim}

\begin{Shaded}
\begin{Highlighting}[]
\KeywordTok{mshapiro.test}\NormalTok{(}\KeywordTok{as.matrix}\NormalTok{(}\KeywordTok{t}\NormalTok{(iris[iris}\OperatorTok{$}\NormalTok{Y}\OperatorTok{==}\StringTok{"setosa"}\NormalTok{,}\DecValTok{1}\OperatorTok{:}\DecValTok{4}\NormalTok{])))}
\end{Highlighting}
\end{Shaded}

\begin{verbatim}
## 
##  Shapiro-Wilk normality test
## 
## data:  Z
## W = 0.95878, p-value = 0.07906
\end{verbatim}

\begin{Shaded}
\begin{Highlighting}[]
\KeywordTok{mshapiro.test}\NormalTok{(}\KeywordTok{as.matrix}\NormalTok{(}\KeywordTok{t}\NormalTok{(iris[iris}\OperatorTok{$}\NormalTok{Y}\OperatorTok{==}\StringTok{"virginica"}\NormalTok{,}\DecValTok{1}\OperatorTok{:}\DecValTok{4}\NormalTok{])))}
\end{Highlighting}
\end{Shaded}

\begin{verbatim}
## 
##  Shapiro-Wilk normality test
## 
## data:  Z
## W = 0.93414, p-value = 0.007955
\end{verbatim}

\begin{enumerate}
\def\labelenumi{\arabic{enumi}.}
\setcounter{enumi}{9}
\tightlist
\item
  Commenter : Ici on rejetterai l'hypothèse de normalité, sauf pour la
  classe setosa
\end{enumerate}

A l'aide de la fonction contenue dans le fichier \texttt{BoxMTest.R} on
réalise le test d'égalité des matrices de variances-covariances.

\begin{Shaded}
\begin{Highlighting}[]
\KeywordTok{source}\NormalTok{(}\StringTok{"BoxMTest.R"}\NormalTok{) }\CommentTok{# Fichier à récupérer sur moodle}
\KeywordTok{BoxMTest}\NormalTok{(iris[,}\DecValTok{1}\OperatorTok{:}\DecValTok{4}\NormalTok{],iris}\OperatorTok{$}\NormalTok{Y)}
\end{Highlighting}
\end{Shaded}

\begin{verbatim}
## ------------------------------------------------
##  MBox Chi-sqr. df P
## ------------------------------------------------
##   146.6632   140.9430          20       0.0000
## ------------------------------------------------
## Covariance matrices are significantly different.
\end{verbatim}

\begin{verbatim}
## $MBox
##   setosa 
## 146.6632 
## 
## $ChiSq
##  setosa 
## 140.943 
## 
## $df
## [1] 20
## 
## $pValue
##       setosa 
## 3.352034e-20
\end{verbatim}

\begin{enumerate}
\def\labelenumi{\arabic{enumi}.}
\setcounter{enumi}{10}
\tightlist
\item
  Commenter : Ici on rejette l'hypothèse d'homogénéité des variances.
  Par conséquent nous ne somme pas sous les conditions d'application du
  test de la MANOVA, on peut alors lui des versions non paramètriques ne
  reposant pas sur l'hypothèse de normalité, comme par exemple de le
  test de Kruskal-Wallis multivarié.
\end{enumerate}

A l'aide de la fonction \texttt{manova} de R tester l'égalité des
espérances des groupes :
\texttt{manova(cbind(X1,X2,X3,X4)\ \textasciitilde{}\ Y,\ data\ =\ iris)}

\begin{Shaded}
\begin{Highlighting}[]
\NormalTok{iris_manova =}\StringTok{ }\KeywordTok{manova}\NormalTok{(}\KeywordTok{cbind}\NormalTok{(X1,X2,X3,X4)}\OperatorTok{~}\NormalTok{Y,}\DataTypeTok{data=}\NormalTok{iris)}
\end{Highlighting}
\end{Shaded}

Obtenir les résumés à partir de la fonction \texttt{summary} appliquée à
l'objet précédent :

\begin{Shaded}
\begin{Highlighting}[]
\KeywordTok{summary}\NormalTok{(iris_manova) }\CommentTok{# compléter}
\end{Highlighting}
\end{Shaded}

\begin{verbatim}
##            Df Pillai approx F num Df den Df    Pr(>F)    
## Y           2 1.1919   53.466      8    290 < 2.2e-16 ***
## Residuals 147                                            
## ---
## Signif. codes:  0 '***' 0.001 '**' 0.01 '*' 0.05 '.' 0.1 ' ' 1
\end{verbatim}

\begin{enumerate}
\def\labelenumi{\arabic{enumi}.}
\setcounter{enumi}{11}
\item
  Commenter : Ici la statistique de test utilisée est la statistique de
  Pillai, une transformation appliquée à la statistique de test suit
  approximativement une loi de Fisher à \(8\) et \(290\) dgegrés de
  liberté. Ici on rejette \(H_0\).
\item
  Aller voir dans l'aide de la fonction \texttt{summary.manova} pour
  modifier la statistique de test utilisée
\end{enumerate}

\begin{Shaded}
\begin{Highlighting}[]
\KeywordTok{help}\NormalTok{(}\StringTok{"summary.manova"}\NormalTok{)}
\end{Highlighting}
\end{Shaded}

\begin{verbatim}
## starting httpd help server ... done
\end{verbatim}

\begin{Shaded}
\begin{Highlighting}[]
\KeywordTok{summary}\NormalTok{(iris_manova,}\StringTok{"Pillai"}\NormalTok{)}
\end{Highlighting}
\end{Shaded}

\begin{verbatim}
##            Df Pillai approx F num Df den Df    Pr(>F)    
## Y           2 1.1919   53.466      8    290 < 2.2e-16 ***
## Residuals 147                                            
## ---
## Signif. codes:  0 '***' 0.001 '**' 0.01 '*' 0.05 '.' 0.1 ' ' 1
\end{verbatim}

\begin{Shaded}
\begin{Highlighting}[]
\KeywordTok{summary}\NormalTok{(iris_manova,}\StringTok{"Wilks"}\NormalTok{)  }
\end{Highlighting}
\end{Shaded}

\begin{verbatim}
##            Df    Wilks approx F num Df den Df    Pr(>F)    
## Y           2 0.023439   199.15      8    288 < 2.2e-16 ***
## Residuals 147                                              
## ---
## Signif. codes:  0 '***' 0.001 '**' 0.01 '*' 0.05 '.' 0.1 ' ' 1
\end{verbatim}

\begin{Shaded}
\begin{Highlighting}[]
\KeywordTok{summary}\NormalTok{(iris_manova,}\StringTok{"Hotelling-Lawley"}\NormalTok{) }
\end{Highlighting}
\end{Shaded}

\begin{verbatim}
##            Df Hotelling-Lawley approx F num Df den Df    Pr(>F)    
## Y           2           32.477   580.53      8    286 < 2.2e-16 ***
## Residuals 147                                                      
## ---
## Signif. codes:  0 '***' 0.001 '**' 0.01 '*' 0.05 '.' 0.1 ' ' 1
\end{verbatim}

\begin{Shaded}
\begin{Highlighting}[]
\KeywordTok{summary}\NormalTok{(iris_manova,}\StringTok{"Roy"}\NormalTok{) }
\end{Highlighting}
\end{Shaded}

\begin{verbatim}
##            Df    Roy approx F num Df den Df    Pr(>F)    
## Y           2 32.192     1167      4    145 < 2.2e-16 ***
## Residuals 147                                            
## ---
## Signif. codes:  0 '***' 0.001 '**' 0.01 '*' 0.05 '.' 0.1 ' ' 1
\end{verbatim}

\begin{enumerate}
\def\labelenumi{\arabic{enumi}.}
\setcounter{enumi}{13}
\tightlist
\item
  Commenter les résultats obtenus : Dans chacun des cas on rejette
  \(H_0\).
\end{enumerate}

Par la suite on va calculer les matrices \(W\) et \(B\) qui pourraient
être utilisées pour récalculer les statistiques de test ci-dessus.

\hypertarget{analyse-factorielle-discriminante-iris-de-fisher}{%
\section{Analyse factorielle discriminante (iris de
Fisher)}\label{analyse-factorielle-discriminante-iris-de-fisher}}

\hypertarget{calcul-des-matrices}{%
\subsection{Calcul des matrices}\label{calcul-des-matrices}}

\begin{enumerate}
\def\labelenumi{\arabic{enumi}.}
\setcounter{enumi}{14}
\tightlist
\item
  Calculer \(V\) la matrice de variance-covariance globale, à partir de
  la fonction \texttt{cov.wt} en utilisant l'option
  \texttt{method = "ML"}. Expliquer à quoi sert cette option.
\end{enumerate}

\begin{Shaded}
\begin{Highlighting}[]
\NormalTok{V =}\StringTok{ }\KeywordTok{cov.wt}\NormalTok{(iris[,}\DecValTok{1}\OperatorTok{:}\DecValTok{4}\NormalTok{],}\DataTypeTok{method =} \StringTok{"ML"}\NormalTok{)}\OperatorTok{$}\NormalTok{cov}
\end{Highlighting}
\end{Shaded}

Attention on prendra garde de récupérer le bon élément de sortie de la
fonction \texttt{cov.wt}, fonction qui ressort une liste contenant entre
autres \texttt{cov}, \texttt{center}, \ldots{}

On calcule les vecteurs des moyennes pour chaque groupe \(\bar{X}_{i}\)
en s'aidant de la fonction \texttt{by}, et en restructurant le résultat
sous forme d'un tableau.

Constituez la matrice \(G\) de centres des classes composée d'une
colonne par variable et d'une ligne par sous-espèce (on rappelle que la
fonction \texttt{t} permet de transposer un tableau)

\begin{Shaded}
\begin{Highlighting}[]
\KeywordTok{by}\NormalTok{(iris[,}\DecValTok{1}\OperatorTok{:}\DecValTok{4}\NormalTok{],iris}\OperatorTok{$}\NormalTok{Y, colMeans)}
\end{Highlighting}
\end{Shaded}

\begin{verbatim}
## iris$Y: setosa
##    X1    X2    X3    X4 
## 5.006 3.428 1.462 0.246 
## ------------------------------------------------------------ 
## iris$Y: versicolor
##    X1    X2    X3    X4 
## 5.936 2.770 4.260 1.326 
## ------------------------------------------------------------ 
## iris$Y: virginica
##    X1    X2    X3    X4 
## 6.588 2.974 5.552 2.026
\end{verbatim}

\begin{Shaded}
\begin{Highlighting}[]
\KeywordTok{simplify2array}\NormalTok{(}\KeywordTok{by}\NormalTok{(iris[,}\DecValTok{1}\OperatorTok{:}\DecValTok{4}\NormalTok{],iris}\OperatorTok{$}\NormalTok{Y, colMeans))}
\end{Highlighting}
\end{Shaded}

\begin{verbatim}
##    setosa versicolor virginica
## X1  5.006      5.936     6.588
## X2  3.428      2.770     2.974
## X3  1.462      4.260     5.552
## X4  0.246      1.326     2.026
\end{verbatim}

\begin{Shaded}
\begin{Highlighting}[]
\NormalTok{G =}\StringTok{ }\KeywordTok{t}\NormalTok{(}\KeywordTok{simplify2array}\NormalTok{(}\KeywordTok{by}\NormalTok{(iris[,}\DecValTok{1}\OperatorTok{:}\DecValTok{4}\NormalTok{],iris}\OperatorTok{$}\NormalTok{Y, colMeans)))}
\end{Highlighting}
\end{Shaded}

\begin{enumerate}
\def\labelenumi{\arabic{enumi}.}
\setcounter{enumi}{15}
\tightlist
\item
  En déduire : \[
  B = \sum_{i=1}^{K}\frac{n_i}{n}(\bar{X}_{i}-\bar{X})(\bar{X}_{i}-\bar{X})^T
  \] où \(\bar{X}_{i}\), \(\bar{X}\) sont respectivement les vecteurs
  colonnes des moyennes intra-classes et de la moyenne globale. Pour
  cela on pourra remarquer que \(B\) est la matrice de covariance des
  centres des classes pondérés par leurs effectifs (penser à
  \texttt{cov.wt} et à son argument \texttt{wt}).
\end{enumerate}

\begin{Shaded}
\begin{Highlighting}[]
\NormalTok{B =}\StringTok{ }\KeywordTok{cov.wt}\NormalTok{(G, }\DataTypeTok{wt =} \KeywordTok{as.vector}\NormalTok{(}\KeywordTok{table}\NormalTok{(iris}\OperatorTok{$}\NormalTok{Y)) , }\DataTypeTok{method =} \StringTok{"ML"}\NormalTok{)}\OperatorTok{$}\NormalTok{cov}
\end{Highlighting}
\end{Shaded}

Calcul de W :

\begin{Shaded}
\begin{Highlighting}[]
\NormalTok{Wi =}\StringTok{ }\KeywordTok{lapply}\NormalTok{(}\KeywordTok{levels}\NormalTok{(iris}\OperatorTok{$}\NormalTok{Y), }\ControlFlowTok{function}\NormalTok{(k)}
  \KeywordTok{cov.wt}\NormalTok{(iris[iris}\OperatorTok{$}\NormalTok{Y}\OperatorTok{==}\StringTok{ }\NormalTok{k,}\DecValTok{1}\OperatorTok{:}\DecValTok{4}\NormalTok{],}\DataTypeTok{method=}\StringTok{"ML"}\NormalTok{)}\OperatorTok{$}\NormalTok{cov) }\CommentTok{# Liste de Wi}
\NormalTok{ni =}\StringTok{ }\KeywordTok{table}\NormalTok{(iris}\OperatorTok{$}\NormalTok{Y)  }\CommentTok{# Vecteur de ni}

\NormalTok{W =}\StringTok{ }\KeywordTok{Reduce}\NormalTok{(}\StringTok{'+'}\NormalTok{,}\KeywordTok{Map}\NormalTok{(}\StringTok{'*'}\NormalTok{,Wi,ni))}\OperatorTok{/}\KeywordTok{sum}\NormalTok{(ni)}
\end{Highlighting}
\end{Shaded}

\begin{enumerate}
\def\labelenumi{\arabic{enumi}.}
\setcounter{enumi}{16}
\tightlist
\item
  Vérifier qu'on retrouve bien : \(V = W + B\)
\end{enumerate}

\begin{Shaded}
\begin{Highlighting}[]
\CommentTok{# Proposer un indicateur synthétique du fait que V = W + B}
\KeywordTok{norm}\NormalTok{(V }\OperatorTok{-}\StringTok{ }\NormalTok{(W }\OperatorTok{+}\StringTok{ }\NormalTok{B)) }
\end{Highlighting}
\end{Shaded}

\begin{verbatim}
## [1] 2.942091e-15
\end{verbatim}

\hypertarget{ruxe9alisation-de-lafd}{%
\subsection{Réalisation de l'AFD}\label{ruxe9alisation-de-lafd}}

On rappelle que dans R l'ACP peut se réaliser à la main comme suit :

\begin{Shaded}
\begin{Highlighting}[]
\KeywordTok{eigen}\NormalTok{(V) }\CommentTok{# Decomposition en valeurs propres}
\end{Highlighting}
\end{Shaded}

\begin{verbatim}
## eigen() decomposition
## $values
## [1] 4.20005343 0.24105294 0.07768810 0.02367619
## 
## $vectors
##             [,1]        [,2]        [,3]       [,4]
## [1,]  0.36138659 -0.65658877 -0.58202985  0.3154872
## [2,] -0.08452251 -0.73016143  0.59791083 -0.3197231
## [3,]  0.85667061  0.17337266  0.07623608 -0.4798390
## [4,]  0.35828920  0.07548102  0.54583143  0.7536574
\end{verbatim}

\begin{Shaded}
\begin{Highlighting}[]
\KeywordTok{eigen}\NormalTok{(V)}\OperatorTok{$}\NormalTok{values}
\end{Highlighting}
\end{Shaded}

\begin{verbatim}
## [1] 4.20005343 0.24105294 0.07768810 0.02367619
\end{verbatim}

\begin{Shaded}
\begin{Highlighting}[]
\NormalTok{ACP=}\KeywordTok{eigen}\NormalTok{(V)}\OperatorTok{$}\NormalTok{vectors}
\NormalTok{c=}\KeywordTok{as.matrix}\NormalTok{(iris[,}\DecValTok{1}\OperatorTok{:}\DecValTok{4}\NormalTok{])}\OperatorTok\NormalTok{ACP[,}\DecValTok{1}\OperatorTok{:}\DecValTok{2}\NormalTok{]}
\KeywordTok{plot}\NormalTok{(c,}\DataTypeTok{col=}\NormalTok{iris}\OperatorTok{$}\NormalTok{Y)}
\end{Highlighting}
\end{Shaded}

\includegraphics{corrige_TP1_files/figure-latex/unnamed-chunk-10-1.pdf}

\begin{Shaded}
\begin{Highlighting}[]
\NormalTok{c =}\StringTok{ }\KeywordTok{as.data.frame}\NormalTok{(c)}
\KeywordTok{names}\NormalTok{(c) <-}\StringTok{ }\KeywordTok{c}\NormalTok{(}\StringTok{"C1"}\NormalTok{,}\StringTok{"C2"}\NormalTok{)}
\NormalTok{c }\OperatorTok\StringTok{ }\KeywordTok{mutate}\NormalTok{(}\DataTypeTok{Y =}\NormalTok{ iris}\OperatorTok{$}\NormalTok{Y) }\OperatorTok\StringTok{ }
\StringTok{  }\KeywordTok{ggplot}\NormalTok{(}\KeywordTok{aes}\NormalTok{(}\DataTypeTok{x =}\NormalTok{ C1, }\DataTypeTok{y =}\NormalTok{ C2, }\DataTypeTok{color =}\NormalTok{ Y, }\DataTypeTok{shape =}\NormalTok{ Y)) }\OperatorTok{+}\StringTok{ }
\StringTok{  }\KeywordTok{geom_point}\NormalTok{()}
\end{Highlighting}
\end{Shaded}

\includegraphics{corrige_TP1_files/figure-latex/unnamed-chunk-10-2.pdf}

\begin{enumerate}
\def\labelenumi{\arabic{enumi}.}
\setcounter{enumi}{17}
\tightlist
\item
  Commenter, quel est le pourcentage d'inertie expliqué par chacun des
  axes ? Pour les deux premiers axes
\end{enumerate}

\begin{Shaded}
\begin{Highlighting}[]
\NormalTok{l =}\StringTok{ }\KeywordTok{eigen}\NormalTok{(V)}\OperatorTok{$}\NormalTok{values}
\NormalTok{l}
\end{Highlighting}
\end{Shaded}

\begin{verbatim}
## [1] 4.20005343 0.24105294 0.07768810 0.02367619
\end{verbatim}

\begin{Shaded}
\begin{Highlighting}[]
\NormalTok{prop_var =}\StringTok{ }\NormalTok{l}\OperatorTok{/}\KeywordTok{sum}\NormalTok{(l) }\CommentTok{# 0.924618723 0.053066483 0.017102610 0.005212184}
\KeywordTok{cumsum}\NormalTok{(prop_var)}
\end{Highlighting}
\end{Shaded}

\begin{verbatim}
## [1] 0.9246187 0.9776852 0.9947878 1.0000000
\end{verbatim}

\begin{enumerate}
\def\labelenumi{\arabic{enumi}.}
\setcounter{enumi}{18}
\tightlist
\item
  Calculer les coordonnées \(d_1\) et \(d_2\) des points projetés sur
  les deux premières composantes discriminantes, sachant qu'en AFD on
  diagonalise la matrice \(V^{-1}B\). Adapter le code pour réaliser
  l'AFD, et commenter les résultats (on rappelle que l'inverse s'obtient
  avec la fonction \texttt{solve} et le produit matriciel avec
  l'opérateur \texttt{\%*\%}) :
\end{enumerate}

\begin{Shaded}
\begin{Highlighting}[]
\NormalTok{M =}\StringTok{ }\KeywordTok{solve}\NormalTok{(V) }\OperatorTok\StringTok{ }\NormalTok{B}
\KeywordTok{eigen}\NormalTok{(M) }\CommentTok{# Decomposition en valeurs propres}
\end{Highlighting}
\end{Shaded}

\begin{verbatim}
## eigen() decomposition
## $values
## [1]  9.698722e-01  2.220266e-01  5.441883e-15 -5.292628e-16
## 
## $vectors
##            [,1]         [,2]       [,3]       [,4]
## [1,]  0.2087418 -0.006531964  0.2106843 -0.8850525
## [2,]  0.3862037 -0.586610553 -0.3962378  0.2969366
## [3,] -0.5540117  0.252561540 -0.4591404  0.2754433
## [4,] -0.7073504 -0.769453092  0.7666797  0.2294377
\end{verbatim}

\begin{Shaded}
\begin{Highlighting}[]
\KeywordTok{eigen}\NormalTok{(M)}\OperatorTok{$}\NormalTok{values}
\end{Highlighting}
\end{Shaded}

\begin{verbatim}
## [1]  9.698722e-01  2.220266e-01  5.441883e-15 -5.292628e-16
\end{verbatim}

\begin{Shaded}
\begin{Highlighting}[]
\NormalTok{AFD=}\KeywordTok{eigen}\NormalTok{(M)}\OperatorTok{$}\NormalTok{vectors}
\NormalTok{d=}\KeywordTok{as.matrix}\NormalTok{(iris[,}\DecValTok{1}\OperatorTok{:}\DecValTok{4}\NormalTok{])}\OperatorTok\NormalTok{AFD[,}\DecValTok{1}\OperatorTok{:}\DecValTok{2}\NormalTok{]}
\KeywordTok{plot}\NormalTok{(d,}\DataTypeTok{col=}\NormalTok{iris}\OperatorTok{$}\NormalTok{Y)}
\end{Highlighting}
\end{Shaded}

\includegraphics{corrige_TP1_files/figure-latex/unnamed-chunk-12-1.pdf}

\begin{Shaded}
\begin{Highlighting}[]
\NormalTok{d =}\StringTok{ }\KeywordTok{as.data.frame}\NormalTok{(d)}
\KeywordTok{names}\NormalTok{(d) <-}\StringTok{ }\KeywordTok{c}\NormalTok{(}\StringTok{"D1"}\NormalTok{,}\StringTok{"D2"}\NormalTok{)}
\NormalTok{d }\OperatorTok\StringTok{ }\KeywordTok{mutate}\NormalTok{(}\DataTypeTok{Y =}\NormalTok{ iris}\OperatorTok{$}\NormalTok{Y) }\OperatorTok\StringTok{ }
\StringTok{  }\KeywordTok{ggplot}\NormalTok{(}\KeywordTok{aes}\NormalTok{(}\DataTypeTok{x =}\NormalTok{ D1, }\DataTypeTok{y =}\NormalTok{ D2, }\DataTypeTok{color =}\NormalTok{ Y, }\DataTypeTok{shape =}\NormalTok{ Y)) }\OperatorTok{+}\StringTok{ }
\StringTok{  }\KeywordTok{geom_point}\NormalTok{()}
\end{Highlighting}
\end{Shaded}

\includegraphics{corrige_TP1_files/figure-latex/unnamed-chunk-12-2.pdf}

Quels est la part de variance de \(d_1\) expliquée par la classe ? De
\(d_2\)

\begin{enumerate}
\def\labelenumi{\arabic{enumi}.}
\setcounter{enumi}{19}
\tightlist
\item
  Reprendre le code précédent en remplaçant \(V^{-1}B\) par \(W^{-1}B\).
\end{enumerate}

\begin{Shaded}
\begin{Highlighting}[]
\KeywordTok{eigen}\NormalTok{(}\KeywordTok{solve}\NormalTok{(V) }\OperatorTok\StringTok{ }\NormalTok{B)}\OperatorTok{$}\NormalTok{vectors }
\end{Highlighting}
\end{Shaded}

\begin{verbatim}
##            [,1]         [,2]       [,3]       [,4]
## [1,]  0.2087418 -0.006531964  0.2106843 -0.8850525
## [2,]  0.3862037 -0.586610553 -0.3962378  0.2969366
## [3,] -0.5540117  0.252561540 -0.4591404  0.2754433
## [4,] -0.7073504 -0.769453092  0.7666797  0.2294377
\end{verbatim}

\begin{Shaded}
\begin{Highlighting}[]
\KeywordTok{eigen}\NormalTok{(}\KeywordTok{solve}\NormalTok{(W) }\OperatorTok\StringTok{ }\NormalTok{B)}\OperatorTok{$}\NormalTok{vectors}
\end{Highlighting}
\end{Shaded}

\begin{verbatim}
##            [,1]         [,2]        [,3]       [,4]
## [1,] -0.2087418 -0.006531964 -0.69871384  0.3910846
## [2,] -0.3862037 -0.586610553  0.02318900 -0.4306213
## [3,]  0.5540117  0.252561540 -0.03791239 -0.4836812
## [4,]  0.7073504 -0.769453092  0.71401953  0.6539653
\end{verbatim}

\begin{Shaded}
\begin{Highlighting}[]
\NormalTok{l =}\StringTok{ }\KeywordTok{eigen}\NormalTok{(}\KeywordTok{solve}\NormalTok{(V) }\OperatorTok\StringTok{ }\NormalTok{B)}\OperatorTok{$}\NormalTok{values }\CommentTok{# lambda}
\KeywordTok{eigen}\NormalTok{(}\KeywordTok{solve}\NormalTok{(W) }\OperatorTok\StringTok{ }\NormalTok{B)}\OperatorTok{$}\NormalTok{values }\CommentTok{# lambda/(1-lambda)}
\end{Highlighting}
\end{Shaded}

\begin{verbatim}
## [1]  3.219193e+01  2.853910e-01 -6.565242e-15  4.298805e-15
\end{verbatim}

\begin{Shaded}
\begin{Highlighting}[]
\NormalTok{l}\OperatorTok{/}\NormalTok{(}\DecValTok{1}\OperatorTok{-}\NormalTok{l)}
\end{Highlighting}
\end{Shaded}

\begin{verbatim}
## [1]  3.219193e+01  2.853910e-01  5.441883e-15 -5.292628e-16
\end{verbatim}

Que dire ? Quel est le lien entre les différents vecteurs propres et
valeurs propres ?

\begin{enumerate}
\def\labelenumi{\arabic{enumi}.}
\setcounter{enumi}{20}
\tightlist
\item
  Comparer les résultats obtenus à ceux obtenus en ACP.
\end{enumerate}

\hypertarget{calcul-des-scores-discriminants}{%
\subsection{Calcul des scores
discriminants}\label{calcul-des-scores-discriminants}}

On souhaite calculer les fonctions de score pour chacun des groupes, ces
fonctions nous serviront ensuite à affecter chaque individu au groupe de
plus grand score (équivalent à la minimisation de la distance de
Mahalanobis).

On rappelle que le calcul des fonctions de score pour chaque groupe
s'effectue comme suit : \[
s_i(x) = \alpha_{i0} + \alpha_{i1}x_1 + \alpha_{i2}x_2 + \alpha_{i3}x_3 + \alpha_{i4}x_4 
\] avec \(\alpha_{i0} = - \bar{X}_i^T W^{-1} \bar{X}_i\) et \[
\begin{pmatrix} \alpha_{i1} \\ \vdots \\ \alpha_{ip} \end{pmatrix} = 2 W^{-1} \bar{X}_i
\]

\begin{enumerate}
\def\labelenumi{\arabic{enumi}.}
\setcounter{enumi}{21}
\tightlist
\item
  Construire le tableau des coefficients :
\end{enumerate}

\begin{longtable}[]{@{}llll@{}}
\toprule
& Setosa & Versicolor & Virginica\tabularnewline
\midrule
\endhead
Constante & \(\alpha_{10}\) & \(\alpha_{20}\) &
\(\alpha_{30}\)\tabularnewline
\(X_1\) & \(\alpha_{11}\) & \(\alpha_{21}\) &
\(\alpha_{31}\)\tabularnewline
\(X_2\) & \(\alpha_{12}\) & \(\alpha_{22}\) &
\(\alpha_{32}\)\tabularnewline
\(X_3\) & \(\alpha_{13}\) & \(\alpha_{23}\) &
\(\alpha_{33}\)\tabularnewline
\(X_4\) & \(\alpha_{14}\) & \(\alpha_{24}\) &
\(\alpha_{34}\)\tabularnewline
\bottomrule
\end{longtable}

Exemple à la main :

\begin{Shaded}
\begin{Highlighting}[]
\NormalTok{i =}\StringTok{ }\DecValTok{1} \CommentTok{# Classe 1 (setosa)}
\KeywordTok{dim}\NormalTok{(G[i,])}
\end{Highlighting}
\end{Shaded}

\begin{verbatim}
## NULL
\end{verbatim}

\begin{Shaded}
\begin{Highlighting}[]
\KeywordTok{dim}\NormalTok{(G[i,,}\DataTypeTok{drop =} \OtherTok{FALSE}\NormalTok{])}
\end{Highlighting}
\end{Shaded}

\begin{verbatim}
## [1] 1 4
\end{verbatim}

\begin{Shaded}
\begin{Highlighting}[]
\NormalTok{Xi <-}\StringTok{ }\KeywordTok{matrix}\NormalTok{(G[i,],}\DecValTok{4}\NormalTok{,}\DecValTok{1}\NormalTok{)}
\CommentTok{# Xi <- t(G[i,,drop = FALSE])}
\OperatorTok{-}\StringTok{ }\KeywordTok{t}\NormalTok{(Xi) }\OperatorTok\StringTok{ }\KeywordTok{solve}\NormalTok{(W) }\OperatorTok\StringTok{ }\NormalTok{Xi  }\CommentTok{# Premier alpha_\{i0\}}
\end{Highlighting}
\end{Shaded}

\begin{verbatim}
##           [,1]
## [1,] -173.8977
\end{verbatim}

\begin{Shaded}
\begin{Highlighting}[]
\DecValTok{2} \OperatorTok{*}\StringTok{ }\KeywordTok{solve}\NormalTok{(W) }\OperatorTok\StringTok{ }\NormalTok{Xi }\CommentTok{# Les 4 autres ! }
\end{Highlighting}
\end{Shaded}

\begin{verbatim}
##         [,1]
## X1  48.04932
## X2  48.13851
## X3 -33.53192
## X4 -35.50696
\end{verbatim}

Mise en production !

\begin{Shaded}
\begin{Highlighting}[]
\NormalTok{alpha=}\KeywordTok{matrix}\NormalTok{(}\DecValTok{0}\NormalTok{,}\DecValTok{5}\NormalTok{,}\DecValTok{3}\NormalTok{)}
\KeywordTok{rownames}\NormalTok{(alpha) =}\StringTok{ }\KeywordTok{c}\NormalTok{(}\StringTok{"intercept"}\NormalTok{,}\StringTok{"X1"}\NormalTok{,}\StringTok{"X2"}\NormalTok{,}\StringTok{"X3"}\NormalTok{,}\StringTok{"X4"}\NormalTok{)}
\KeywordTok{colnames}\NormalTok{(alpha) =}\StringTok{ }\KeywordTok{levels}\NormalTok{(iris}\OperatorTok{$}\NormalTok{Y)}
\ControlFlowTok{for}\NormalTok{ (i }\ControlFlowTok{in} \DecValTok{1}\OperatorTok{:}\DecValTok{3}\NormalTok{) \{}
\NormalTok{  barXi=}\KeywordTok{matrix}\NormalTok{(G[i,],}\DecValTok{4}\NormalTok{,}\DecValTok{1}\NormalTok{) }\CommentTok{# centres de Xj pour Y=i}
\NormalTok{  alpha[}\DecValTok{1}\NormalTok{,i]=}\OperatorTok{-}\KeywordTok{t}\NormalTok{(barXi)}\OperatorTok\KeywordTok{solve}\NormalTok{(W)}\OperatorTok\NormalTok{barXi}
\NormalTok{  alpha[}\DecValTok{2}\OperatorTok{:}\DecValTok{5}\NormalTok{,i]=}\DecValTok{2}\OperatorTok{*}\KeywordTok{solve}\NormalTok{(W)}\OperatorTok\NormalTok{barXi}
\NormalTok{\}}
\NormalTok{alpha}
\end{Highlighting}
\end{Shaded}

\begin{verbatim}
##               setosa versicolor   virginica
## intercept -173.89767 -146.43672 -210.754506
## X1          48.04932   32.03716   25.399692
## X2          48.13851   14.43369    7.520979
## X3         -33.53192   10.63561   26.054173
## X4         -35.50696   13.13108   43.018598
\end{verbatim}

Aide : Dans l'AFD, la notion de score est liée au calcul de la règle de
décision. Une observation \(x = (x_1, x_2, x_3, x_p)\) sera affectée au
groupe avec le score \(s_i(x)\) maximal.

Rappel : \[
\hat{y} = \arg\min_i (x - \bar{X}_i)^T W^{-1} (x - \bar{X}_i) 
\] Ce calcul revient à maximiser
\(2 x^T W^{-1} \bar{X}_i - \bar{X}_i^T W^{-1}\bar{X}_i\).

\begin{enumerate}
\def\labelenumi{\arabic{enumi}.}
\setcounter{enumi}{22}
\tightlist
\item
  Calculer les scores des individus à partir de cette règle (simple
  calcul matriciel, on pourra rajouter une colonne de 1 à la matrice des
  données à l'aide de la fonction \texttt{cbind})
\end{enumerate}

Exemple à la main pour le calcul des scores

\begin{Shaded}
\begin{Highlighting}[]
\NormalTok{alpha[}\DecValTok{1}\NormalTok{,}\DecValTok{1}\NormalTok{] }\OperatorTok{+}\StringTok{ }\KeywordTok{sum}\NormalTok{(iris[}\DecValTok{1}\NormalTok{,}\DecValTok{1}\OperatorTok{:}\DecValTok{4}\NormalTok{] }\OperatorTok{*}\StringTok{ }\NormalTok{alpha[}\DecValTok{2}\OperatorTok{:}\DecValTok{5}\NormalTok{,}\DecValTok{1}\NormalTok{]) }\CommentTok{# Score pour l'individu 1 dans la classe 1}
\end{Highlighting}
\end{Shaded}

\begin{verbatim}
## [1] 185.5926
\end{verbatim}

\begin{Shaded}
\begin{Highlighting}[]
\KeywordTok{c}\NormalTok{(}\DecValTok{1}\NormalTok{,}\KeywordTok{as.matrix}\NormalTok{(iris[}\DecValTok{1}\NormalTok{,}\DecValTok{1}\OperatorTok{:}\DecValTok{4}\NormalTok{])) }\OperatorTok\StringTok{ }\NormalTok{alpha[,}\DecValTok{1}\NormalTok{, drop =}\StringTok{ }\OtherTok{FALSE}\NormalTok{]}
\end{Highlighting}
\end{Shaded}

\begin{verbatim}
##        setosa
## [1,] 185.5926
\end{verbatim}

\begin{Shaded}
\begin{Highlighting}[]
\KeywordTok{c}\NormalTok{(}\DecValTok{1}\NormalTok{,}\KeywordTok{as.matrix}\NormalTok{(iris[}\DecValTok{1}\NormalTok{,}\DecValTok{1}\OperatorTok{:}\DecValTok{4}\NormalTok{])) }\OperatorTok\StringTok{ }\NormalTok{alpha}
\end{Highlighting}
\end{Shaded}

\begin{verbatim}
##        setosa versicolor virginica
## [1,] 185.5926    84.9868 -9.813089
\end{verbatim}

\begin{Shaded}
\begin{Highlighting}[]
\KeywordTok{head}\NormalTok{(}\KeywordTok{as.matrix}\NormalTok{(}\KeywordTok{cbind}\NormalTok{(}\DecValTok{1}\NormalTok{,iris[}\DecValTok{1}\OperatorTok{:}\DecValTok{4}\NormalTok{])) }\OperatorTok\StringTok{ }\NormalTok{alpha)}
\end{Highlighting}
\end{Shaded}

\begin{verbatim}
##        setosa versicolor  virginica
## [1,] 185.5926   84.98680  -9.813089
## [2,] 151.9135   71.36252 -18.653517
## [3,] 155.2845   66.77827 -24.834677
## [4,] 138.9593   64.25830 -22.915909
## [5,] 185.6015   83.22645 -11.600960
## [6,] 202.1018  106.18833  17.235182
\end{verbatim}

Score individu : simple produit matriciel

\begin{Shaded}
\begin{Highlighting}[]
\NormalTok{s=}\KeywordTok{as.matrix}\NormalTok{(}\KeywordTok{cbind}\NormalTok{(}\DecValTok{1}\NormalTok{,iris[,}\DecValTok{1}\OperatorTok{:}\DecValTok{4}\NormalTok{]))}\OperatorTok\NormalTok{alpha}
\NormalTok{s[}\DecValTok{1}\OperatorTok{:}\DecValTok{10}\NormalTok{,]}
\end{Highlighting}
\end{Shaded}

\begin{verbatim}
##         setosa versicolor  virginica
##  [1,] 185.5926   84.98680  -9.813089
##  [2,] 151.9135   71.36252 -18.653517
##  [3,] 155.2845   66.77827 -24.834677
##  [4,] 138.9593   64.25830 -22.915909
##  [5,] 185.6015   83.22645 -11.600960
##  [6,] 202.1018  106.18833  17.235182
##  [7,] 153.2034   68.83796 -18.963173
##  [8,] 172.6206   81.40328 -10.499739
##  [9,] 123.0749   53.90057 -32.105461
## [10,] 156.9248   72.55635 -19.597861
\end{verbatim}

\begin{Shaded}
\begin{Highlighting}[]
\KeywordTok{names}\NormalTok{(}\KeywordTok{which.max}\NormalTok{(s[}\DecValTok{150}\NormalTok{,]))}
\end{Highlighting}
\end{Shaded}

\begin{verbatim}
## [1] "virginica"
\end{verbatim}

\begin{Shaded}
\begin{Highlighting}[]
\NormalTok{Ypredit =}\StringTok{ }\KeywordTok{apply}\NormalTok{(s, }\DecValTok{1}\NormalTok{, }\ControlFlowTok{function}\NormalTok{(x) }\KeywordTok{names}\NormalTok{(}\KeywordTok{which.max}\NormalTok{(x)))}
\end{Highlighting}
\end{Shaded}

\begin{enumerate}
\def\labelenumi{\arabic{enumi}.}
\setcounter{enumi}{23}
\tightlist
\item
  En déduire le classement de chacun des individus à partir de ces
  scores (en utilisant de façon appropriée les fonctions \texttt{apply}
  et \texttt{which.max}) :
\end{enumerate}

On affecte l'individu dans la colonne où le score est le plus élevé :

\begin{Shaded}
\begin{Highlighting}[]
\NormalTok{Ypredit=}\KeywordTok{levels}\NormalTok{(iris}\OperatorTok{$}\NormalTok{Y)[}\KeywordTok{apply}\NormalTok{(s,}\DecValTok{1}\NormalTok{,which.max)]}
\NormalTok{Ypredit[}\DecValTok{1}\OperatorTok{:}\DecValTok{10}\NormalTok{]}
\end{Highlighting}
\end{Shaded}

\begin{verbatim}
##  [1] "setosa" "setosa" "setosa" "setosa" "setosa" "setosa" "setosa" "setosa"
##  [9] "setosa" "setosa"
\end{verbatim}

\begin{Shaded}
\begin{Highlighting}[]
\KeywordTok{table}\NormalTok{(Ypredit)}
\end{Highlighting}
\end{Shaded}

\begin{verbatim}
## Ypredit
##     setosa versicolor  virginica 
##         50         49         51
\end{verbatim}

\begin{Shaded}
\begin{Highlighting}[]
\KeywordTok{table}\NormalTok{(}\DataTypeTok{Y =}\NormalTok{ iris}\OperatorTok{$}\NormalTok{Y,Ypredit) }\CommentTok{# Matrice de confusion}
\end{Highlighting}
\end{Shaded}

\begin{verbatim}
##             Ypredit
## Y            setosa versicolor virginica
##   setosa         50          0         0
##   versicolor      0         48         2
##   virginica       0          1        49
\end{verbatim}

\begin{Shaded}
\begin{Highlighting}[]
\NormalTok{TBC =}\StringTok{ }\KeywordTok{mean}\NormalTok{(iris}\OperatorTok{$}\NormalTok{Y }\OperatorTok{==}\StringTok{ }\NormalTok{Ypredit)}
\NormalTok{TBC}
\end{Highlighting}
\end{Shaded}

\begin{verbatim}
## [1] 0.98
\end{verbatim}

\begin{Shaded}
\begin{Highlighting}[]
\NormalTok{TMC =}\StringTok{ }\KeywordTok{mean}\NormalTok{(iris}\OperatorTok{$}\NormalTok{Y }\OperatorTok{!=}\StringTok{ }\NormalTok{Ypredit)}
\NormalTok{TMC}
\end{Highlighting}
\end{Shaded}

\begin{verbatim}
## [1] 0.02
\end{verbatim}

\end{document}
